\begin{opinion}

\noindent Estudiante: Adrian Chang Alcover\\

Durante los últimos meses el estudiante Adrian Chang Alcover ha venido desarrollando la tesis de diploma ``Generación Automática de Horarios como un Problema de Satisfacción de Restricciones'' que constituye uno de los requisitos para convertirse en Licenciado en Ciencias de la Computación. Durante ese período Adrian fue capaz de leer bibliografía actualizada, aprender tecnologías modernas, utilizar patrones de diseño y buenas prácticas de programación.

Trabajó de manera muy independiente y en la mayoría de las situaciones fue capaz de resolver los problemas a los que se enfrentó con mucha destreza lo que demuestra la solidez de los conocimientos adquiridos durante la carrera. Integró conocimientos y habilidades de diferentes asignaturas y disciplinas de varios años.

Creemos que está listo para obtener su título de Licenciado en Ciencias de la Computación y enfrentarse a la vida profesional como una persona muy capaz y con sobrados conocimientos técnicos y una elevada calidad humana.\\

\noindent La Habana 30 de mayo de 2014\\
Lic. Pedro Quintero Rojas\\
Profesor Asistente del Departamento de Inteligencia Artificial y Sistemas Distribuidos de la facultad de Matemática y Computación de la Universidad de La Habana

\vfill
\begin{flushleft}
\underline{\hspace{140pt}}\hfill \underline{\hspace{140pt}}\\
Lic. Pedro Quintero Rojas \hfill Dr. Yuri Quintana Pacheco \hspace*{7pt}
\end{flushleft}
\end{opinion}