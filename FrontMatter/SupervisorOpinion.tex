\begin{opinion}

El presente trabajo tiene como objetivo general proponer una solución para la generación automática de un horario universitario de clases, basado en la modelación de este problema como un problema de satisfacción de restricciones.

El estudiante realizó un estudio organizado y profundo de los temas necesarios para brindar una propuesta de solución eficaz al problema que enfrentó, fundamentalmente el problema de satisfacción de restricciones.

El trabajo realizado por el estudiante fue llevado a cabo con un alto grado de independencia, desde la modelación del problema hasta la implementación y el diseño experimental. Es de destacar además, que el trabajo se enmarca dentro de un proyecto más general de automatización de un sistema de horarios, por lo que el estudiante tuvo que ser capaz de interactuar con un sistema implementado por otras personas, e incorporar como habilidad el trabajo en equipo.

La solución propuesta por el estudiante cumple con todos los requerimientos de un trabajo de diploma. Además, como parte del trabajo fueron diseñados un número importante de experimentos para sustentar las distintas variantes de solución consideradas.

Por lo anterior expuesto propongo para el estudiante la calificación de excelente (5 puntos).\\

\noindent Dr. Yuri Quintana Pacheco\\
Facultad de Matemática y Ciencia de la Computación\\
Universidad de la Habana

\vfill
\begin{flushleft}
\underline{\hspace{140pt}}\hfill \underline{\hspace{140pt}}\\
Lic. Pedro Quintero Rojas \hfill Dr. Yuri Quintana Pacheco \hspace*{7pt}
\end{flushleft}
\end{opinion}