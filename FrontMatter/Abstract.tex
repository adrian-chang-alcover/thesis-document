\begin{abstract}

En este trabajo se presenta una posible soluci\'on para automatizar la generaci\'on de un horario universitario de clases, para ello se model\'o como un Problema de Satisfacci\'on de Restricciones. Este trabajo se integra a un Sistema de Gesti\'on de Horarios desarrollado en la facultad de Matem\'atica y Computaci\'on de la Universidad de La Habana, de donde se obtiene toda la informaci\'on necesaria para el horario, adem\'as de las restricciones o preferencias impuestas por cada profesor de la facultad. Se implementaron diferentes heur\'isticas y t\'ecnicas de filtrado que se suelen usar para este tipo de problemas. Se  hace un an\'alisis a los resultados obtenidos de haber aplicado los distintos algoritmos y t\'ecnicas al problema en cuesti\'on. Se probaron distintas instancias del problema de los horarios mediante la modificaci\'on y la adici\'on de nuevas restricciones. Adem\'as se implementaron otros dos problemas y se le aplicaron los algoritmos en aras de comparar la calidad de dichos algoritmos frente a problemas diferentes. La estrategia propuesta produce mejores soluciones que el método manual, en cuanto a la calidad de los cronogramas obtenidos y a la cantidad de tiempo empleado para generarlos.
\end{abstract}