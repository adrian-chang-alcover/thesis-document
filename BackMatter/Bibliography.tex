% This file was created with JabRef 2.7b.
% Encoding: UTF-8

\bibliographystyle{babplain-uh}
\bibliography{Bibliography}

\begin{thebibliography}{99}

\bibitem{Carter's summary} MW Carter*, G Laporte†, and JW Chinneck, “A General Examination Scheduling
System,” Interfaces, vol. 24, no. 3, pp. 109-120, The Institute of Management Sciences,
May-June 1994. *Department of Industrial Engineering, University of
Toronto, Ontario, Canada; Ecole des Hautes Etudes Commerciales de Montreal,
Quebec, Canada; and Department of Systems and Computer Engineering, Carleton
University, Ottawa, Ontario, Canada

\bibitem{Bloomfield and McSharry says} SD Bloomfield and MM McSharry, “Preferential Course Scheduling,” Interfaces, vol.9, no. 4, pp. 24-31, The Institute of Management Sciences, August 1979. School of Business, Oregon State University, Corvallis, USA

\bibitem{Romero} BP Romero, “Examination Scheduling in a Large Engineering School: A
Computer-Assisted Participative Procedure,” Interfaces, vol. 12, no. 2, pp. 17-23, The
Institute of Management Sciences, April 1982. Industrial Organization Department,
Industrial Engineering Technical School, Madrid, Spain

\bibitem{survey of University of Nottingham} EK Burke, DG Elliman, PH Ford, and RF Weare, “Exam Timetabling in British
Universities – A Survey,” in the Practice and Theory of Automated Timetabling, ed. EK
Burke and P Ross, pp. 76-90, Springer-Verlag (Lecture Notes in Computer Science),
1996. Department of Computer Science, University of Nottingham, UK

\end{thebibliography}