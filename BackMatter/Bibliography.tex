% This file was created with JabRef 2.7b.
% Encoding: UTF-8

\bibliographystyle{babplain-uh}
\bibliography{Bibliography}

\begin{thebibliography}{99}

\bibitem{TB Cooper and JH Kingston}TB Cooper and JH Kingston, “The Complexity of Timetable Construction Problems,”
in the Practice and Theory of Automated Timetabling, ed. EK Burke and P Ross,
pp. 283-295, Springer-Verlag (Lecture Notes in Computer Science), 1996. Basser
Department of Computer Science, University of Sydney, Australia

\bibitem{Carter's summary} MW Carter*, G Laporte†, and JW Chinneck, “A General Examination Scheduling
System,” Interfaces, vol. 24, no. 3, pp. 109-120, The Institute of Management Sciences,
May-June 1994. *Department of Industrial Engineering, University of
Toronto, Ontario, Canada; Ecole des Hautes Etudes Commerciales de Montreal,
Quebec, Canada; and Department of Systems and Computer Engineering, Carleton
University, Ottawa, Ontario, Canada

\bibitem{Bloomfield and McSharry says} SD Bloomfield and MM McSharry, “Preferential Course Scheduling,” Interfaces, vol.9, no. 4, pp. 24-31, The Institute of Management Sciences, August 1979. School of Business, Oregon State University, Corvallis, USA

\bibitem{Romero} BP Romero, “Examination Scheduling in a Large Engineering School: A
Computer-Assisted Participative Procedure,” Interfaces, vol. 12, no. 2, pp. 17-23, The
Institute of Management Sciences, April 1982. Industrial Organization Department,
Industrial Engineering Technical School, Madrid, Spain

\bibitem{survey of University of Nottingham} EK Burke, DG Elliman, PH Ford, and RF Weare, “Exam Timetabling in British
Universities – A Survey,” in the Practice and Theory of Automated Timetabling, ed. EK
Burke and P Ross, pp. 76-90, Springer-Verlag (Lecture Notes in Computer Science),
1996. Department of Computer Science, University of Nottingham, UK

\bibitem{VA Bardadym} VA Bardadym, “Computer-Aided School and University Timetabling: The New
Wave,” in the Practice and Theory of Automated Timetabling, ed. EK Burke and P Ross,
pp. 22-45, Springer-Verlag (Lecture Notes in Computer Science), 1996. International
Soros Science Education Program, International Renaissance Foundation,
Kiev, Ukraine

\bibitem{MW Carter} MW Carter, “A Survey of Practical Applications of Examination Timetabling Algorithms,”
Operations Research, vol. 34, no. 2, pp. 193-201, Operations Research Society
of America, March-April 1986. University of Toronto, Ontario, Canada

\bibitem{MW Carter and G Laporte} MW Carter* and G Laporte†, “Recent Developments in Practical Examination Timetabling,”
in the Practice and Theory of Automated Timetabling, ed. EK Burke and P
Ross, pp. 3-21, Springer-Verlag (Lecture Notes in Computer Science), 1996.
*Department of Industrial Engineering, University of Toronto, Ontario, Canada;
and ††Ecole des Hautes Etudes Commerciales de Montreal, Quebec, Canada

\bibitem{JH Kingston} JH Kingston*, VA Bardadym, and MW Carter, “Bibliography on the Practice and
Theory of Automated Timetabling,” in ftp://ftp.cs.usyd.edu.au/jeff/-
timetabling/timetabling.bib.gz. *Basser Department of Computer Science,
University of Sydney, Australia

\bibitem{D Abramson and J Abela}D Abramson and J Abela, T Arani*, and V Lotfi†, “A Three Phased Approach to
Final Exam Scheduling,” IIE Transactions, vol. 21, no. 1, pp. 86-96, IIE, Australia,
March 1989. (technical report)

\bibitem{D Corne and P Ross and HL Fang}D Corne, P Ross, and HL Fang, “Fast Practical Evolutionary Timetabling,” LectureNotes in Computer Science, vol. 865 (Artificial Intelligence and Simulation of
Behaviour (AISB) Workshop on Evolutionary Computing, University of Leeds, UK,
11th-13th April 1994), pp. 251-263, Springer-Verlag, 1994. Department of Artificial
Intelligence, University of Edinburgh, UK

\bibitem{B Paechter* and A Cumming* and H Luchian}B Paechter*, A Cumming*, H Luchian†, and M Petriuc‡, “Two Solutions to the General Timetable Problem using Evolutionary Methods,” proceedings of the IEEE
Conference on Evolutionary Computation 1994. *Computer Studies Department,
Napier University, Edinburgh, Scotland, UK; †Faculty of Computer Science, Al I
Cuza University of Iasi, Romania; and ‡Technical University of Iasi, Romania

\bibitem{EK Burke and DG Elliman and RF Weare 1}EK Burke, DG Elliman, and RF Weare, “The Automation of the Timetabling Process in Higher Education,” Journal of Educational Technology Systems, vol. 23, no. 4,
pp. 257-266, Baywood Publishing Company, 1995. Department of Computer Science,
University of Nottingham, UK

\bibitem{EK Burke and DG Elliman and RF Weare 2}EK Burke, DG Elliman, and RF Weare, “A Hybrid Genetic Algorithm for Highly Constrained Timetabling Problems,” 6th International Conference on Genetic Algorithms
(ICGA’95, Pittsburgh, USA, 15th-19th July 1995), Morgan Kaufmann, San Francisco,
CA, USA. Department of Computer Science, University of Nottingham, UK

\bibitem{EK Burke and DG Elliman and RF Weare 3}EK Burke, DG Elliman, and RF Weare, “Specialised Recombinative Operators for Timetabling Problems,” proceedings of the AISB (Artificial Intelligence and Simulation
of Behaviour) Workshop on Evolutionary Computing (University of Sheffield, UK, 3rd-7th
April 1995), pp. 75-85, Springer-Verlag, 1995. Department of Computer Science,
University of Nottingham, UK

\bibitem{EK Burke and JP Newall and RF Weare}EK Burke, JP Newall, and RF Weare, “A Memetic Algorithm for University Exam
Timetabling,” in the Practice and Theory of Automated Timetabling, ed. EK Burke and
P Ross, pp. 241-250, Springer-Verlag (Lecture Notes in Computer Science), 1996.
Department of Computer Science, University of Nottingham, UK

\bibitem{B Paechter and A Cumming and MG Norman and H Luchian}B Paechter*, A Cumming*, MG Norman†, and H Luchian‡, “Extensions to a Memetic Timetabling System,” in the Practice and Theory of Automated Timetabling,
ed. EK Burke and P Ross, pp. 251-265, Springer-Verlag (Lecture Notes in Computer
Science), 1996. *Computer Studies Department, Napier University, Edinburgh,
Scotland, UK; †Makespan Ltd, Edinburgh, UK; and ‡Faculty of Computer Science,
Al I Cuza University of Iasi, Romania

\bibitem{J Thompson and KA Dowsland}J Thompson and KA Dowsland, “Variants of Simulated Annealing for the Examination
Timetabling Problem,” Annals of Operations Research, 1995. European Business
Management School, University of Wales at Swansea, UK

\bibitem{J Thompson and KA Dowsland 2}J Thompson and KA Dowsland, “General Cooling Schedules for a Simulated
Annealing based Timetabling System,” in the Practice and Theory of Automated Timetabling,
ed. EK Burke and P Ross, pp. 345-363, Springer-Verlag (Lecture Notes in Computer Science), 1996. European Business Management School, University of Wales at Swansea, UK

\end{thebibliography}