% This file was created with JabRef 2.7b.
% Encoding: UTF-8

\bibliographystyle{babplain-uh}
\bibliography{Bibliography}

\begin{thebibliography}{99}

\bibitem{TB Cooper and JH Kingston}TB Cooper and JH Kingston, “The Complexity of Timetable Construction Problems,”
in the Practice and Theory of Automated Timetabling, ed. EK Burke and P Ross,
pp. 283-295, Springer-Verlag (Lecture Notes in Computer Science), 1996. Basser
Department of Computer Science, University of Sydney, Australia

\bibitem{Carter's summary} MW Carter*, G Laporte†, and JW Chinneck, “A General Examination Scheduling
System,” Interfaces, vol. 24, no. 3, pp. 109-120, The Institute of Management Sciences,
May-June 1994. *Department of Industrial Engineering, University of
Toronto, Ontario, Canada; Ecole des Hautes Etudes Commerciales de Montreal,
Quebec, Canada; and Department of Systems and Computer Engineering, Carleton
University, Ottawa, Ontario, Canada

\bibitem{Bloomfield and McSharry says} SD Bloomfield and MM McSharry, “Preferential Course Scheduling,” Interfaces, vol.9, no. 4, pp. 24-31, The Institute of Management Sciences, August 1979. School of Business, Oregon State University, Corvallis, USA

\bibitem{Romero} BP Romero, “Examination Scheduling in a Large Engineering School: A
Computer-Assisted Participative Procedure,” Interfaces, vol. 12, no. 2, pp. 17-23, The
Institute of Management Sciences, April 1982. Industrial Organization Department,
Industrial Engineering Technical School, Madrid, Spain

\bibitem{survey of University of Nottingham} EK Burke, DG Elliman, PH Ford, and RF Weare, “Exam Timetabling in British
Universities – A Survey,” in the Practice and Theory of Automated Timetabling, ed. EK
Burke and P Ross, pp. 76-90, Springer-Verlag (Lecture Notes in Computer Science),
1996. Department of Computer Science, University of Nottingham, UK

\bibitem{VA Bardadym} VA Bardadym, “Computer-Aided School and University Timetabling: The New
Wave,” in the Practice and Theory of Automated Timetabling, ed. EK Burke and P Ross,
pp. 22-45, Springer-Verlag (Lecture Notes in Computer Science), 1996. International
Soros Science Education Program, International Renaissance Foundation,
Kiev, Ukraine

\bibitem{MW Carter} MW Carter, “A Survey of Practical Applications of Examination Timetabling Algorithms,”
Operations Research, vol. 34, no. 2, pp. 193-201, Operations Research Society
of America, March-April 1986. University of Toronto, Ontario, Canada

\bibitem{MW Carter and G Laporte} MW Carter* and G Laporte†, “Recent Developments in Practical Examination Timetabling,”
in the Practice and Theory of Automated Timetabling, ed. EK Burke and P
Ross, pp. 3-21, Springer-Verlag (Lecture Notes in Computer Science), 1996.
*Department of Industrial Engineering, University of Toronto, Ontario, Canada;
and ††Ecole des Hautes Etudes Commerciales de Montreal, Quebec, Canada

\bibitem{JH Kingston} JH Kingston*, VA Bardadym, and MW Carter, “Bibliography on the Practice and
Theory of Automated Timetabling,” in ftp://ftp.cs.usyd.edu.au/jeff/-
timetabling/timetabling.bib.gz. *Basser Department of Computer Science,
University of Sydney, Australia

\bibitem{D Abramson and J Abela}D Abramson and J Abela, T Arani*, and V Lotfi†, “A Three Phased Approach to
Final Exam Scheduling,” IIE Transactions, vol. 21, no. 1, pp. 86-96, IIE, Australia,
March 1989. (technical report)

\end{thebibliography}