\begin{conclusions}

El diseño de un horario universitario de clases es una tarea bastante compleja que debe ser automatizada. M\'as aun cuando intervienen las preferencias y restricciones de cientos de profesores pertenecientes al centro. Adem\'as ante la ocurrencia de eventos inesperados se debe rediseñar el horario lo m\'as pronto posible y con el m\'inimo de cambios necesarios para continuar con el cumplimiento del resto de las actividades. Trabajos realizados por otras universidades son muy espec\'ificos de las mismas, no incluyen aspectos que se consideran importantes en este trabajo y generalmente son dif\'iciles de adaptar y extender.

Para automatizar el proceso de generaci\'on de horarios se model\'o este como un Problema de Satisfacci\'on de Restricciones. Se mencionaron las ventajas que aporta dicha modelaci\'on. Fueron implementados algoritmos, heur\'isticas y t\'ecnicas de filtrado que se suelen usar para este tipo de problemas. Se analizaron dos instancias de este problema mediante la adici\'on de nuevas restricciones y se pudo observar c\'omo las heur\'isticas y t\'ecnicas en algunos casos funcionan mejor en que otros. Adem\'as se implement\'o el problema de las n-reinas y un problema de criptograf\'ia sencillo con el objetivo de comparar los algoritmos implementados frente a diferentes problemas.

Se explic\'o la interacci\'on con el Sistema de Gesti\'on de Horarios y c\'omo se obtiene toda la informaci\'on necesaria para generar un horario. Para simplificar el problema se propuso como metodolog\'ia generar el horario por semanas.

Para futuros trabajos se recomienda implementar t\'ecnicas de filtrado para CSPs m\'as avanzadas como \emph{path consistency}. Por otro lado, el grafo de restricciones para este problema es completo, es decir, cada variable est\'a relacionada con el resto de las variables, esto tiene un impacto negativo fundamentalmente para las t\'ecnicas de filtrado \emph{forward checking} y \emph{arc consistency}. Ser\'ia de gran ayuda si se lograra obtener un grafo de restricciones menos denso que el actual. Otro aspecto en el que se pudiera trabajar es en la exportaci\'on de la funci\'on de felicidad del sistema y la optimizaci\'on de esta para que pueda ser usada en contextos donde se requiera m\'ultiples evaluaciones de la misma.

\end{conclusions}