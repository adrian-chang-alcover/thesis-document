\begin{conclusions}

El diseño de un horario universitario de clases es una tarea bastante compleja que debe ser automatizada. M\'as aun cuando intervienen las preferencias y restricciones de cientos de profesores pertenecientes al centro. Adem\'as ante la ocurrencia de eventos inesperados se debe rediseñar el horario lo m\'as pronto posible y con el m\'inimo de cambios necesarios para continuar con el cumplimiento del resto de las actividades. Trabajos realizados por otras universidades son muy espec\'ificos de las mismas, no incluyen aspectos que se consideran importantes en este trabajo y generalmente son dif\'iciles de adaptar y extender.

Para automatizar el proceso de generaci\'on de horarios se model\'o este como un Problema de Satisfacci\'on de Restricciones. Se mencionaron las ventajas que aporta dicha modelaci\'on. Fueron implementados algoritmos, heur\'isticas y t\'ecnicas de filtrado que se suelen usar para este tipo de problemas.

\end{conclusions}