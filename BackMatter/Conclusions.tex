\begin{conclusions}
En el presente trabajo se propuso una metodología para 
el proceso de minería de opinión sobre \emph{Twitter}, basada en la hipótesis 
de que existen clasificadores que se adaptan mejor a la estructura del
problema y por lo tanto devuelven mejores resultados que el resto.
La propuesta final presentada en los resultados cumple con 
los objetivos marcados en este trabajo.

Como parte del mismo se analizaron las distintas fases del 
proceso de minería de opinión y su aplicación a \emph{Twitter}.
Además se construyeron diccionarios para el trabajo con jerga, emoticones
y corrección ortográfica en español.

Para validar los experimentos se construyó un corpus de \emph{tweets} en español
clasificados en objetivo, positivo, negativo o neutro. 
Sobre el mismo se analizaron diferentes propuestas,
determinando el mejor algoritmo en cada una de las fases. Finalmente
se presenta una propuesta completa para el proceso de minería de opinión.

% A partir de los experimentos realizados con los diferentes
% preprocesamientos propuestos se determinó que no son 
% influyentes las variaciones en el proceso de normalización
% del texto. Aunque es necesario realizar un proceso básico 
% que al menos elimine los signos de puntuación y separe correctamente 
% las palabras. Algunos de los preprocesamientos probados
% provocan errores en la clasificación como la 
% corrección ortográfica, que provoca que un gran número de
% términos que no aparecen en los diccionarios sean modificados 
% y cambien el sentido del documento.
\end{conclusions}

\begin{recomendations}
La investigación discutida en la tesis puede generalizarse
extendiendo la metodología propuesta a otros idiomas. Esto 
provoca cambios fundamentalmente en la etapa de preprocesamiento
donde se utilizan varios diccionarios y procesos que dependen del idioma 
utilizado~\footnote{diccionario de emoticones, diccionario de jerga, corrección 
ortográfica}. Para algunos de estos diccionarios se brinda una primera 
aproximación para idioma inglés, que no fue utilizada en este trabajo.
La metodología propuesta de manera general si es independiente del 
del idioma en el que estén los datos.

Para enriquecer los resultados propuestos pueden realizarse 
experimentos con nuevos algoritmos o variantes de los 
propuestos. En el trabajo se propone la utilización de algoritmos
estándares sin una parametrización exhaustiva de los mismos,
para poder realizar un análisis general de la propuesta. Por lo tanto 
pueden realizarse mejoras a los mismos.

El tamaño del corpus tiene una gran influencia sobre la precisión
de los resultados obtenidos. En consecuencia se propone continuar incrementando 
el corpus propuesto, utilizando la aplicación creada para la 
clasificación de mensajes. 

La tesis se concentra en la clasificación de mensajes sin tener en cuenta 
información semántica de los mismos. Una propuesta a largo 
plazo puede incluir clasificadores que utilicen información adicional
como los emoticones, relaciones de sinonimia, aparición de signos de puntuación 
especiales y conocimiento sobre los términos~\footnote{Puede utilizarse información
sobre la polaridad del término o el papel gramatical que ocupa~(adjetivo, sustantivo, verbo)}.

El proyecto resultante de este trabajo está concebido
como parte de un proyecto general que se está desarrollando en el
grupo de investigación. Es por tanto necesario integrar
de forma efectiva la metodología propuesta en el
entorno de análisis de información sobre
\emph{Twitter} que se encuentra en desarrollo.
\end{recomendations}