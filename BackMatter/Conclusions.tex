\begin{conclusions}
En el presente trabajo se propuso una metodología para 
el proceso de minería de opinión sobre \emph{Twitter}, basada en la hipótesis 
de que existen clasificadores que se adaptan mejor a la estructura del
problema y por lo tanto devuelven mejores resultados que el resto.
La propuesta final presentada en los resultados cumple con 
los objetivos marcados en este trabajo.

Como parte del mismo se analizaron las distintas fases del 
proceso de minería de opinión y su aplicación a \emph{Twitter}.
Además se construyeron diccionarios para el trabajo con jerga, emoticones
y corrección ortográfica en español.

Para validar los experimentos se construyó un corpus de \emph{tweets} en español
clasificados en objetivo, positivo, negativo o neutro. 
Sobre el mismo se analizaron diferentes propuestas,
determinando el mejor algoritmo en cada una de las fases. Finalmente
se presenta una propuesta completa para el proceso de minería de opinión.

% A partir de los experimentos realizados con los diferentes
% preprocesamientos propuestos se determinó que no son 
% influyentes las variaciones en el proceso de normalización
% del texto. Aunque es necesario realizar un proceso básico 
% que al menos elimine los signos de puntuación y separe correctamente 
% las palabras. Algunos de los preprocesamientos probados
% provocan errores en la clasificación como la 
% corrección ortográfica, que provoca que un gran número de
% términos que no aparecen en los diccionarios sean modificados 
% y cambien el sentido del documento.
\end{conclusions}