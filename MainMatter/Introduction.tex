\begin{introduction}

La planificación de actividades, entiéndase por sucesión de tareas a cumplir
con el objetivo de lograr un fin, siempre ha sido un problema recurrente en la
vida cotidiana. Más ahora con el desarrollo que existe en la ciencia y la tecnología,
donde el ahorro de recursos y en particular el tiempo, juega un papel fundamental.

Este problema está presente en todas las esferas de la vida. En el mundo empresarial,
especial importancia tiene la planificación de reuniones, donde se debe respetar un
orden cronológico y las actividades de las personas involucradas no debe verse afectada
por la reunión en cuestión. En el ámbito industrial la fabricación de una determinada
pieza requiere de varios procesos, en cada proceso intervienen personal y una maquinaria
específica, además debe realizarse en un tiempo determinado. Una buena planificación
de las actividades o procesos involucrados determina el buen cumplimiento del objetivo
final.

En la educación se necesita una detallada planificación de todos los turnos de clases que se
impartirán a lo largo del presente curso. Por ejemplo, todas las asignaturas deben cubrir el total de horas
especificadas, debe tenerse en cuenta que un grupo no tenga asignado más de un turno de clase
a la vez, as\'i como un profesor no debe tener planificado más de un turno de clase al mismo tiempo.

La educación universitaria no está exenta de este problema, donde cada nuevo curso necesita del
diseño de un horario con ciertas particularidades que lo diferencian de otros niveles de educación.
Hay que tener en cuenta que en una universidad existen varias facultades, en algunas facultades se
estudian más de una carrera, además turnos en diferentes facultades pueden compartir un recurso común
perteneciente a la universidad en general, sea un salón especial, d\'igase un teatro, o un profesor
que imparta clases en varias facultades.

\section*{Motivación}

Actualmente la facultad de Matemática y Computación de la Universidad de La Habana, centro donde 
se enmarca este trabajo, no cuenta con una herramienta que realice o ayude en la confección del horario.
Por otro lado, trabajos realizados por otras universidades son muy espec\'ificos de las mismas, no incluyen aspectos que se consideran importantes en este trabajo y generalmente son dif\'iciles de adaptar y extender.

Se desear\'ia crear una herramienta que automatice el proceso de diseño de un horario de clases para la universidad.
Además, contar con una herramienta propia permitirá en un futuro modificarla o agregarle nuevas
funcionalidades sin problemas mayores.

Esta herramienta formará parte de un trabajo mucho más elaborado y completo que viene desarrollando
el profesor Lic. Pedro Quintero Rojas, perteneciente a la facultad antes mencionada, quien a su vez
es tutor de la presente tesis. Su trabajo consiste en un sitio web capaz de gestionar el horario,
modificarlo e informar a todos los involucrados, además de almacenar las preferencias o condiciones
de cada profesor, con el objetivo de generar un horario que satisfaga en la medida de lo posible
las necesidades de cada profesor.

La terminación y puesta en funcionamiento de esta herramienta aliviar\'ia la carga de trabajo del
personal destinado a esta engorrosa tarea, permiti\'endole dedicarse a la realización de otras
actividades de igual importancia.

\section*{Problema}

La facultad de Matemática y Computación de la Universidad de La Habana, tiene ciertas
características que lo distinguen del resto de las facultades. En ella se estudian dos carreras,
cuenta con pocas aulas y muchas de ellas son pequeñas. Adem\'as posee cientos de profesores, cada uno
impone condiciones o preferencias que quisieran se tuvieran en cuenta a la hora de diseñar el horario.

Además, toda planificación a largo plazo puede verse afectada por la ocurrencia inesperada de eventos,
d\'igase fenómenos meteorológicos o reuniones de última hora. Es necesario poder ajustarse a las nuevas
condiciones y rediseñar el horario lo más rápido posible para continuar con el buen cumplimiento del
resto de las actividades.

Por todos estos problemas que se pueden presentar, la planificación de un horario es una tarea sumamente 
compleja incluso si se realiza por un grupo de personas. Ser\'ia de mucha ayuda contar con una herramienta
que automatice este proceso.

\section*{Objetivo}

El objeto de investigación de esta tesis lo constituye el horario de una facultad, en particular la facultad de Matem\'atica y Computaci\'on de la Universidad de La Habana, junto con todas las entidades involucradas, como son los profesores, los grupos, las asignaturas, etc.

Se parte de la hipótesis de que teniendo toda la información necesaria, entidades involucradas y la
planificación de los turnos de clases de cada asignatura por semana, llamada P1, es posible generar
automáticamente un horario que satisfaga en la medida de lo posible todas las condiciones y preferencias
impuestas por cada profesor de la facultad. Para ello se modelar\'a el proceso como un Problema de Satisfacci\'on de Restricciones.


\subsection*{Objetivo General}

``Crear un programa que automatice la generación de un horario universitario de clases model\'andolo como un Problema de Satisfacci\'on de Restricciones.''

\subsection*{Objetivos específicos}

\begin{itemize}
	\item Investigar sobre el estado del arte de la generación de horarios, conocido en la literatura
		como \emph{Timetabling}.
	\item Interactuar con el sistema de gestión de horarios, propuesto por el profesor Lic. Pedro
		Quintero Rojas.
	\item Modelar nuestro problema como un Problema de Satisfacción de Restricciones o CSP (\emph{Constraints 
		Satisfaction Problem}).
	\item Aplicar heurísticas y las principales técnicas de filtrado que se utilizan en Problemas de
		Satisfacción de Restricciones.
	\item Analizar los resultados obtenidos y comparar cu\'al técnica se comporta mejor para este tipo de
		problemas.
\end{itemize}

\section*{Propuesta de Solución y Contribuciones}

\section*{Organización de la tesis}

El contenido de la tesis está estructurado de la siguiente forma. El cap\'itulo 1 aborda el estado del arte sobre la generaci\'on autom\'atica de horarios (\emph{timetabling}). Da una definici\'on del problema y explica las principales diferencias entre un horario de clases y un horario de ex\'amenes. Se analizan los resultados arrojados por una encuesta realizada a universidades brit\'anicas acerca de la automatizacion del diseño de un horario. Por \'ultimo se mencionan las t\'ecnicas con que usualmente se resuelven estos problemas y se propone una posible soluci\'on para modelar los datos del problema.

El cap\'itulo 2 aborda el tema de los Problemas de Satisfacci\'on de Restricciones. Comienza brindando una definici\'on de los mismos y luego da a conocer un algoritmo de b\'usqueda recursivo. Expone heur\'isticas de b\'usqueda y t\'ecnicas de filtrado que pueden ser utilizadas en el algoritmo de b\'usqueda recursivo. Tambi\'en se expone c\'omo m\'etodos de b\'usqueda local pueden dar buenos resultados para este tipo de problemas. Por \'ultimo se hace referencia a trabajos realizados en este campo de investigaci\'on 

El cap\'itulo 3 propone una posible soluci\'on al problema de generar un horario universitario de clases model\'andolo como un Problema de Satisfacci\'on de Restricciones. En una primera parte se explica como se interactua con un Sistema de Gesti\'on de Horarios implementado en la facultad antes mencionada. Luego se explica c\'omo se model\'o este problema como un CSP. Por \'ultimo se exponen las implementaciones de los algoritmos y t\'ecnicas vistas en el cap\'itulo 2.

La tesis finaliza presentando las conclusiones obtenidas como resultado de la investigación además de las recomendaciones para trabajos futuros.

Cerrando el trabajo se muestran las referencias bibliográficas consultadas durante la investigación, que ofrecen una mejor perspectiva y que complementan el trabajo.

\end{introduction}
