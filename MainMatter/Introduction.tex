\begin{introduction}

Internet es una plataforma para el intercambio de datos
cada vez más utilizada donde los usuarios, entre otras
acciones, comparten e intercambian opiniones. Con el tiempo, el número 
de usuarios, así como el volumen de información existente, se ha incrementado
considerablemente dificultando el análisis de la
misma. La facilidad del acceso a Internet y la evolución 
de la dinámica de creación de contenidos provoca que la utilicen
cada vez más personas para expresar sus ideas.

La Web, el espacio más visible de Internet, evolucionó de una plataforma fundamentalmente informativa
donde los usuarios consumen la información generada por los suministradores~(instituciones,
agencias de noticias, etc.) a un entorno colaborativo donde los propios usuarios producen
y comparten la información. Este modelo colaborativo es conocido como la Web 2.0. 
Este modo de interactuar ha crecido aceleradamente en los
últimos años debido a la cantidad de información producida. Cualquier individuo con acceso
a la red es un potencial generador de información, muchas veces en forma de opinión.
Debido a la libertad y velocidad con que se genera, además del hecho
de que procede de fuentes informales~(páginas personales, \emph{blogs}),
generalmente, la información presente en la web carece de estructura
y es propensa a errores sintácticos.

Una de las expresiones características de la web 2.0 son las redes
sociales, donde los usuarios 
se comunican a través de la web. En estos espacios de comunicación los
usuarios expresan sus experiencias u opiniones
personales sobre diversos temas, siendo ellos mismos
los generadores de contenido y gestores de la información~\cite{Bollen}.
Dentro de las redes sociales, el \emph{microblogging} ha tenido un espacio 
preponderante propulsado por el auge de los dispositivos y la  
conectividad móvil.
Esta forma de comunicación es más sencilla de
producir, distribuir y consumir. El \emph{microblogging} se basa en la
difusión o publicación de textos breves~(usualmente sobre los 140 caracteres) que
pueden ser distribuidos, generados y consumidos desde
dispositivos móviles, \emph{emails}, mensajería instantánea o directamente en la web. 
Este fenómeno es muy utilizado por los
usuarios para expresar una opinión, informar de una noticia o hacer referencia 
a otros sitios de la web~\cite{Bollen}.

Entre las redes de \emph{microblogging} más conocidas se encuentra \emph{Twitter}
\footnote{\href{http://twitter.com}{http://twitter.com}} 
que hace su aparición en
el año 2006, creándose con él esta nueva forma de comunicación en Internet~\cite{Merlo2010}.
El servicio ha tenido gran
auge, pues a finales del año 2007 se escribieron 500.000 mensajes por trimestre, en el 2008
eran 100 millones los mensajes escritos por trimestre, ya en 2009 se escribieron 2.000 millones de
mensajes por trimestre y para los primeros tres meses del 2010 se habían 
escrito 4.000 millones de mensajes~\cite{Merlo2010}.

%TODO gráfica

Desde su creación \emph{Twitter} se ha considerado una fuente de información para conocer la opinión de
gran cantidad de personas sobre un tema o un producto~\cite{Bollen}. Dado el volumen de información 
existente en este espacio no es posible
realizar esta tarea sin la ayuda de sistemas computacionales que realicen el análisis de forma automática.
Es por ello que se ha realizado un gran número de trabajos de minería de texto y minería de opinión
sobre \emph{Twitter}~\cite{Bollen, Pang}, con el objetivo de extraer información global 
de la red para valorar diversos temas.

%TODO cambiar este párrafo
Conocer la valencia de una opinión es un factor importante para realizar tareas
con impacto social, económico o político tales como: el análisis de
mercado para las empresas, calcular los índices de sa\-tis\-fac\-ción de un producto,
estado de opinión de un país, entre otros. \emph{Twitter} es un entorno 
propicio para realizar trabajos donde se deseen reconocer las opiniones de los usuarios. 

\section*{Motivación}

La minería de texto es una línea de investigación en la inteligencia 
artificial. Se considera muy importante dado que gran cantidad de
información, sobre todo en la web, se encuentra en texto.
En el marco del grupo de Inteligencia Artificial de la 
Universidad de la Habana se desarrolla el proyecto Medialab con
el objetivo de establecer el espacio o marco de las investigaciones y desarrollos
relacionados con recursos y redes sociales en entornos digitales.
Este proyecto considera a \emph{Twitter} como un 
espacio de investigación donde pueden realizarse grandes proyectos 
de análisis de información. Uno de estos proyectos cuenta con una 
tesis de análisis de influencia que se orquesta con los trabajos en 
desarrollo de detección de idioma y detección de tópico. 
Es la intención de este trabajo sobre minería de opinión contribuir a dicho
proyecto.

% Desde el punto de vista académico, la minería de opinión presenta desafíos
% interesantes pues requiere de la aplicación de técnicas de diversas
% áreas de la ciencia de la computación tales como procesamiento del
% lenguaje natural, recuperación de información, reconocimiento de patrones y clasificación. El
% gran volumen de datos no estructurados disponible en las redes sociales
% exige el diseño de algoritmos eficientes para su procesamiento.
% Brinda entonces un marco propicio para el desarrollo de investigaciones
% con acceso a una cantidad de datos sin precedentes.

\section*{Problema}

Conocer la valencia de una opinión es un factor importante para realizar tareas
con impacto social, económico o político tales como: el análisis de
mercado para las empresas, calcular los índices de sa\-tis\-fac\-ción de un producto,
estado de opinión de un país, entre otros.
Un lugar ideal para recoger estas opiniones son las redes sociales, donde las personas 
expresan su opinión espontáneamente. 
\emph{Twitter}, como máximo exponente del \emph{microblogging} en Internet, se ha
convertido en
una manera eficiente de consultar noticias, mantenerse en contacto con los
amigos, entre otras actividades cotidianas. Precisamente estos mensajes de
texto que circulan por \emph{Twitter} constituyen elementos determinantes en el
análisis de la opinión que circula en la red.
El volumen de información almacenado en
estas redes, debido a su popularidad, no es procesable por seres humanos.
Y por tanto se requiere un proceso de automatización que simplifique la tarea. 

%\section{Definición del problema}
%  Se desea determinar de manera automática en un mensaje de texto de \emph{Twitter}, 
%  \emph{Tweet}, si se expresa una opinión
%  o no. En el caso de que se exprese una opinión reconocer si esta es positiva, negativa
%  o neutra.

%\section*{Definición de los datos}
%\label{sec:data_definition}
%  La entrada al problema está formada por un conjunto de mensajes procedentes de \emph{Twitter}.
%  Estos mensajes están escritos de manera informal por lo que pueden presentar irregularidades
%  en su escritura que dificultan la clasificación como: 
%  \begin{itemize}
%   \item Errores ortográficos.
%   \item Letras repetidas.
%   \item Utilización de jerga.
%   \item Utilización de emoticones.
%   \item Inadecuado uso de los signos de puntuación.
%   \item Palabras en varios idiomas.
%   \item Referencias a sitios, URLs.
%   \item Marcas procedentes de \emph{Twitter}~(usuario,tema, otros).
%  \end{itemize}

\section*{Hipótesis y Objetivo}

El objeto de investigación de esta tesis
lo constituye la red social \emph{Twitter},
estableciendo su campo de acción alrededor
de la información
producida por sus usuarios de esta red y las
técnicas necesarias para identificar la misma
como opiniones así como la valencia de las mismas.

La hipótesis plantea que conociendo las características
particulares de la información proveniente de \emph{Twitter}
es posible realizar un procesamiento de la misma que,
utilizando algoritmos de aprendizaje supervisado, 
permita identificar de un conjunto de técnicas cuáles 
se adaptan mejor a las características del problema. 
  
  \subsection*{Objetivo General}
  
  ``Proponer una metodología completa para el desarrollo del proceso
minería de opinión sobre \emph{Twitter}''

  \subsection*{Objetivos específicos}
\begin{itemize}
  \item Estudiar el estado del arte relacionado con el problema
  y las investigaciones anteriores 
  desarrolladas en el grupo de investigación.
  \item Identificar las etapas fundamentales de un proceso de 
  minería de opinión con el fin de adecuarlas al entorno específico
  donde serán aplicadas.
 \item Identificar los elementos susceptibles de ser preprocesados
 y que puedan incidir en el proceso de minería de opinión.
 \item Identificar un conjunto de algoritmos de aprendizaje apropiados
 para detectar las opiniones y su valencia.
 \item Crear un corpus de mensajes de \emph{Twitter} que permita realizar 
 experimentos cuyos resultados arrojen indicios para conformar una metodología.
\end{itemize}

\section*{Propuesta de Solución y Contribuciones}
% 
% Se propone una herramienta para el análisis de opinión en \emph{Tweets} a
% partir de la normalización del los mensajes. La herramienta realizará de forma 
% automática el preprocesamiento del texto y la clasificación del mismo.
% Ambos procesos son configurables por el usuario.
% 
% El preprocesamiento se realiza sobre los mensajes para obtener el texto en 
% lenguaje natural, donde se hayan eliminado los elementos que producen ruido.
% Los clasificadores generalmente se basan en el reconocimiento de estructuras
% o patrones comunes en los mensajes que pertenecen a la misma clase. La existencia
% de elementos de ruido~(palabras mal escritas, etiquetas, emoticones) 
% perturba a los clasificadores, pues proveen de un contenido que no deseado al clasificar
% y pueden hacer parecer diferentes a mensajes que se desea reconocer como semejantes.
% 
% % El modelo de preprocesamiento propuesto se divide en 7 etapas independientes
% % que pueden activarse según los requerimientos del usuario:
% %       
% %       \begin{itemize}
% %        \item Eliminar las etiquetas de \emph{Twitter}.
% %        \item Separar los emoticones del texto.
% %        \item Traducir la jerga.
% %        \item Suprimir las letras repetidas.
% %        \item Eliminar las \emph{stop words}.
% %        \item Aplicar corrección ortográfica.
% %        \item Hacer \emph{stemming}.
% %       \end{itemize}
% 
% Una vez eliminados los elementos de ruido, el mensaje resultante es procesado
% por dos clasificadores en serie. El primero de estos clasificadores determina
% si el mensaje es subjetivo u objetivo; y el segundo determina si un mensaje subjetivo
% es positivo, negativo o neutro.
% 
% En la figura~\ref{fig:clas} se muestra un esquema general del proceso de clasificación.
% 
% Finalmente, la investigación realizada para la implementación del sistema se 
% presenta en esta tesis, la cual se estructuró en cinco capítulos que abordan 
% los diferentes aspectos del trabajo realizado.

%--------------------
Una metodología para el proceso de minería de opinión en \emph{Twitter} 
seguiría las etapas clásicas de representación y normalización de 
los datos. En la clasificación se propone la utilización de dos etapas de
en serie. La primera etapa donde se clasifica
en Objetivo o Subjetivo y la segunda etapa donde se
clasifica en Positivo, Negativo o Neutro, automatizando completamente 
el proceso de clasificación.

El preprocesamiento se realiza sobre los mensajes para obtener el texto en una 
representación común donde se hayan eliminado los elementos que producen ruido.
Los clasificadores generalmente se basan en el reconocimiento de estructuras
o patrones comunes en los mensajes que pertenecen a la misma clase. La existencia
de elementos de ruido~(palabras mal escritas, etiquetas, emoticones) 
perturba a los clasificadores pues los provee de un contenido no deseado, que al clasificar,
provoque que se considere diferentes a mensajes semejantes.

Una vez eliminados los elementos de ruido, el mensaje resultante es procesado
por dos clasificadores en serie. El primero de estos clasificadores determina
si el mensaje es subjetivo u objetivo; y el segundo determina si un mensaje subjetivo
es positivo, negativo o neutro.

% 		\begin{figure}\label{clas}
% 		\begin{center}
% 		\includegraphics[width= 0.7\columnwidth]{Graphics/propuesta}
% 		\end{center}
% 		\caption{Esquema general del proceso de minería de opinión}
% 		\end{figure}

Una propuesta final de la metodología consistiría en la combinación de preprocesamiento y 
clasificadores con los que se logre una mejor clasificación. Además, es necesaria la
creación de un corpus con el que evaluar los resultados obtenidos en la clasificación.

%------------------------
        
\section*{Organización de la tesis}
El contenido de la tesis está estructurado de la siguiente forma. En el 
capítulo 2 se presentan los conceptos de minería de textos, minería de opinión
y sentimiento~(MOS) en \emph{Twitter} y la revisión de trabajos referentes al tema.   
 
En el capítulo 3 se formula la propuesta de metodología para el proceso de minería
de opinión y se describe cada una de sus etapas. Para ello se realiza un 
análisis sobre la estructura de los \emph{tweets} y se propone una normalización del texto.
También se proponen varias soluciones para reducir las dimensiones del 
problema y se presentan los clasificadores seleccionados para determinar 
la aparición de opiniones y la polaridad de las mismas. 

En el capítulo 6 se describe la estructura y compilación del corpus desarrollado
para idioma español. Además se discuten los resultados obtenidos para cada uno de los 
clasificadores presentados en el capítulo anterior y se comprueba la 
validez de la hipótesis planteada.

La tesis finaliza presentando las conclusiones obtenidas como resultado de
la investigación además de las recomendaciones para trabajos futuros. 

Cerrando el trabajo se muestran las referencias
bibliográficas consultadas durante la investigación, que ofrecen una mejor
perspectiva y que complementan el trabajo.

%Anexos?

\end{introduction}
