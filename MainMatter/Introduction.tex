\begin{introduction}

La planificación de actividades, entiendase por sucesión de tareas a cumplir
con el objetivo de lograr un fin, siempre ha sido un problema recurrente en nuestra
vida cotidiana. Más ahora con el desarrollo que existe en la ciencia y la tecnología,
donde el ahorro de recursos y en particular el tiempo, juega un papel fundamental.

Este problema está presente en todas las esferas de nuestra vida. En el mundo empresarial,
especial importancia tiene la planificación de reuniones, donde se debe respetar un
orden cronológico y las actividades de las personas involucradas no debe verse afectada
por la reunión en cuestión. En el ámbito industrial la fabricación de una determinada
pieza requiere de varios procesos, en cada proceso intervienen personal y una maquinaria
específica, además debe realizarse en un tiempo determinado. Una buena planificación
de las actividades o procesos involucrados determina el buen cumplimiento del objetivo
final.

En la educación se necesita una detallada planificación de todos los turnos de clases que se
impartirán a lo largo del presente curso. Todas las asignaturas deben cubrir el total de horas
especificadas, debe tenerse en cuenta que un grupo no tenga asignado más de un turno de clase
a la vez, asi como un profesor no debe tener planificado más de un turno de clase al mismo tiempo.

La educación universitaria no está exenta de este problema, donde cada nuevo curso necesita del
diseño de un horario con ciertas particularidades que lo diferencian de otros niveles de educación.
Hay que tener en cuenta que en una universidad existen varias facultades, en algunas facultades se
estudian más de una carrera, además turnos en diferentes facultades pueden compartir un recurso común
perteneciente a la universidad en general, sea un salón especial, digase un teatro, o un profesor
que imparta clases en varias facultades.

\section*{Motivación}

Actualmente la facultad de Matemática y Computación de la Universidad de La Habana, centro donde 
se enmarca este trabajo, no cuenta con una herramienta que realize o ayude en la confección del horario.
Trabajos realizados por otras universidades no satisfacen completamente nuestras necesidades.
Además contar con una herramienta propia permitirá en un futuro modificarla o agregarle nuevas
funcionalidades sin problemas mayores.

Esta herramienta formará parte de un trabajo mucho más elaborado y completo que viene desarrollando
el profesor Lic. Pedro Quintero Rojas, perteneciente a la facultad antes mencionada, quien a su vez
es tutor de la presente tesis. Su trabajo consiste en un sitio web capaz de gestionar el horario,
modificarlo e informar a todos los involucrados, además de almacenar las preferencias o condiciones
de cada profesor, con el objetivo de generar un horario que satisfaga en la medida de lo posible
las necesidades de cada profesor.

La terminación y puesta en funcionamiento de esta herramienta aliviaria la carga de trabajo del
personal destinado a esta engorrosa tarea, permitiendole dedicarse a la realización de otras
tareas de igual importancia.

\section*{Problema}

Conocer la valencia de una opinión es un factor importante para realizar tareas
con impacto social, económico o político tales como: el análisis de
mercado para las empresas, calcular los índices de sa\-tis\-fac\-ción de un producto,
estado de opinión de un país, entre otros.
Un lugar ideal para recoger estas opiniones son las redes sociales, donde las personas 
expresan su opinión espontáneamente. 
\emph{Twitter}, como máximo exponente del \emph{microblogging} en Internet, se ha
convertido en
una manera eficiente de consultar noticias, mantenerse en contacto con los
amigos, entre otras actividades cotidianas. Precisamente estos mensajes de
texto que circulan por \emph{Twitter} constituyen elementos determinantes en el
análisis de la opinión que circula en la red.
El volumen de información almacenado en
estas redes, debido a su popularidad, no es procesable por seres humanos.
Y por tanto se requiere un proceso de automatización que simplifique la tarea. 

%\section{Definición del problema}
%  Se desea determinar de manera automática en un mensaje de texto de \emph{Twitter}, 
%  \emph{Tweet}, si se expresa una opinión
%  o no. En el caso de que se exprese una opinión reconocer si esta es positiva, negativa
%  o neutra.

%\section*{Definición de los datos}
%\label{sec:data_definition}
%  La entrada al problema está formada por un conjunto de mensajes procedentes de \emph{Twitter}.
%  Estos mensajes están escritos de manera informal por lo que pueden presentar irregularidades
%  en su escritura que dificultan la clasificación como: 
%  \begin{itemize}
%   \item Errores ortográficos.
%   \item Letras repetidas.
%   \item Utilización de jerga.
%   \item Utilización de emoticones.
%   \item Inadecuado uso de los signos de puntuación.
%   \item Palabras en varios idiomas.
%   \item Referencias a sitios, URLs.
%   \item Marcas procedentes de \emph{Twitter}~(usuario,tema, otros).
%  \end{itemize}

\section*{Hipótesis y Objetivo}

El objeto de investigación de esta tesis
lo constituye la red social \emph{Twitter},
estableciendo su campo de acción alrededor
de la información
producida por sus usuarios de esta red y las
técnicas necesarias para identificar la misma
como opiniones así como la valencia de las mismas.

La hipótesis plantea que conociendo las características
particulares de la información proveniente de \emph{Twitter}
es posible realizar un procesamiento de la misma que,
utilizando algoritmos de aprendizaje supervisado, 
permita identificar de un conjunto de técnicas cuáles 
se adaptan mejor a las características del problema. 
  
  \subsection*{Objetivo General}
  
  ``Proponer una metodología completa para el desarrollo del proceso
minería de opinión sobre \emph{Twitter}''

  \subsection*{Objetivos específicos}
\begin{itemize}
  \item Estudiar el estado del arte relacionado con el problema
  y las investigaciones anteriores 
  desarrolladas en el grupo de investigación.
  \item Identificar las etapas fundamentales de un proceso de 
  minería de opinión con el fin de adecuarlas al entorno específico
  donde serán aplicadas.
 \item Identificar los elementos susceptibles de ser preprocesados
 y que puedan incidir en el proceso de minería de opinión.
 \item Identificar un conjunto de algoritmos de aprendizaje apropiados
 para detectar las opiniones y su valencia.
 \item Crear un corpus de mensajes de \emph{Twitter} que permita realizar 
 experimentos cuyos resultados arrojen indicios para conformar una metodología.
\end{itemize}

\section*{Propuesta de Solución y Contribuciones}
% 
% Se propone una herramienta para el análisis de opinión en \emph{Tweets} a
% partir de la normalización del los mensajes. La herramienta realizará de forma 
% automática el preprocesamiento del texto y la clasificación del mismo.
% Ambos procesos son configurables por el usuario.
% 
% El preprocesamiento se realiza sobre los mensajes para obtener el texto en 
% lenguaje natural, donde se hayan eliminado los elementos que producen ruido.
% Los clasificadores generalmente se basan en el reconocimiento de estructuras
% o patrones comunes en los mensajes que pertenecen a la misma clase. La existencia
% de elementos de ruido~(palabras mal escritas, etiquetas, emoticones) 
% perturba a los clasificadores, pues proveen de un contenido que no deseado al clasificar
% y pueden hacer parecer diferentes a mensajes que se desea reconocer como semejantes.
% 
% % El modelo de preprocesamiento propuesto se divide en 7 etapas independientes
% % que pueden activarse según los requerimientos del usuario:
% %       
% %       \begin{itemize}
% %        \item Eliminar las etiquetas de \emph{Twitter}.
% %        \item Separar los emoticones del texto.
% %        \item Traducir la jerga.
% %        \item Suprimir las letras repetidas.
% %        \item Eliminar las \emph{stop words}.
% %        \item Aplicar corrección ortográfica.
% %        \item Hacer \emph{stemming}.
% %       \end{itemize}
% 
% Una vez eliminados los elementos de ruido, el mensaje resultante es procesado
% por dos clasificadores en serie. El primero de estos clasificadores determina
% si el mensaje es subjetivo u objetivo; y el segundo determina si un mensaje subjetivo
% es positivo, negativo o neutro.
% 
% En la figura~\ref{fig:clas} se muestra un esquema general del proceso de clasificación.
% 
% Finalmente, la investigación realizada para la implementación del sistema se 
% presenta en esta tesis, la cual se estructuró en cinco capítulos que abordan 
% los diferentes aspectos del trabajo realizado.

%--------------------
Una metodología para el proceso de minería de opinión en \emph{Twitter} 
seguiría las etapas clásicas de representación y normalización de 
los datos. En la clasificación se propone la utilización de dos etapas de
en serie. La primera etapa donde se clasifica
en Objetivo o Subjetivo y la segunda etapa donde se
clasifica en Positivo, Negativo o Neutro, automatizando completamente 
el proceso de clasificación.

El preprocesamiento se realiza sobre los mensajes para obtener el texto en una 
representación común donde se hayan eliminado los elementos que producen ruido.
Los clasificadores generalmente se basan en el reconocimiento de estructuras
o patrones comunes en los mensajes que pertenecen a la misma clase. La existencia
de elementos de ruido~(palabras mal escritas, etiquetas, emoticones) 
perturba a los clasificadores pues los provee de un contenido no deseado, que al clasificar,
provoque que se considere diferentes a mensajes semejantes.

Una vez eliminados los elementos de ruido, el mensaje resultante es procesado
por dos clasificadores en serie. El primero de estos clasificadores determina
si el mensaje es subjetivo u objetivo; y el segundo determina si un mensaje subjetivo
es positivo, negativo o neutro.

% 		\begin{figure}\label{clas}
% 		\begin{center}
% 		\includegraphics[width= 0.7\columnwidth]{Graphics/propuesta}
% 		\end{center}
% 		\caption{Esquema general del proceso de minería de opinión}
% 		\end{figure}

Una propuesta final de la metodología consistiría en la combinación de preprocesamiento y 
clasificadores con los que se logre una mejor clasificación. Además, es necesaria la
creación de un corpus con el que evaluar los resultados obtenidos en la clasificación.

%------------------------
        
\section*{Organización de la tesis}
El contenido de la tesis está estructurado de la siguiente forma. En el 
capítulo 2 se presentan los conceptos de minería de textos, minería de opinión
y sentimiento~(MOS) en \emph{Twitter} y la revisión de trabajos referentes al tema.   
 
En el capítulo 3 se formula la propuesta de metodología para el proceso de minería
de opinión y se describe cada una de sus etapas. Para ello se realiza un 
análisis sobre la estructura de los \emph{tweets} y se propone una normalización del texto.
También se proponen varias soluciones para reducir las dimensiones del 
problema y se presentan los clasificadores seleccionados para determinar 
la aparición de opiniones y la polaridad de las mismas. 

En el capítulo 6 se describe la estructura y compilación del corpus desarrollado
para idioma español. Además se discuten los resultados obtenidos para cada uno de los 
clasificadores presentados en el capítulo anterior y se comprueba la 
validez de la hipótesis planteada.

La tesis finaliza presentando las conclusiones obtenidas como resultado de
la investigación además de las recomendaciones para trabajos futuros. 

Cerrando el trabajo se muestran las referencias
bibliográficas consultadas durante la investigación, que ofrecen una mejor
perspectiva y que complementan el trabajo.

%Anexos?

\end{introduction}
