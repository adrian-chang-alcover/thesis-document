\chapter{Propuesta de Solución}

En este cap\'itulo se presenta una posible soluci\'on al problema de generar un horario universitario de clases de forma autom\'atica, para ello se modelar\'a como un Problema de Satisfacci\'on de Restricciones. Del Sistema de Gesti\'on de Horarios se explicar\'a la modelaci\'on de las entidades involucradas, la funci\'on de felicidad del sistema y un API para interactuar con el mismo. Luego se mostrar\'a c\'omo se definieron las variables y los dominios de valores de las mismas, las restricciones del problema y la estructura del grafo de restricciones. Por \'ultimo se examinar\'an los algoritmos y t\'ecnicas utilizadas para darle soluci\'on al problema.

\section{Interacci\'on con el Sistema de Gesti\'on de Horarios}

Como ya se hab\'ia mencionado este trabajo formar\'a parte de un Sistema de Gesti\'on de Horarios desarrollado por el profesor Lic. Pedro Quintero Rojas perteneciente a la facultad de Matem\'atica y Computaci\'on de la Universidad de La Habana. Dicho sistema consiste en un sitio web implementado en Ruby on Rails.

El sistema contiene toda la informaci\'on necesaria para generar un horario de clases. Tambi\'en muestra una vista actualizada del horario vigente y las personas autorizadas pueden modificar turnos planificados y reservar aulas de manera transparente. En un futuro se desea implementar una herramienta para tel\'efonos celulares donde se pueda interactuar con el sistema y de alguna forma avisar a las personas involucradas cuando se cambie un turno de clase. 

\subsection{Modelaci\'on de las entidades involucradas}

Para generar un horario de clases de un semestre para una facultad es necesario tener de antemano toda la informaci\'on referente. Se necesita conocer:

\begin{itemize}

\item Las carreras que se estudian en la facultad.
\item Los años docentes que tiene cada carrera.
\item Cantidad de grupos por año docente.
\item Por cada año docente conocer las asignaturas que se deben dar en el semestre en cuesti\'on.
\item Por cada semana del semestre conocer la cantidad y el tipo de turno que se deben impartir de cada asignatura.
\item Los profesores que imparten cada asignatura.
\item Capacidad y tipo de locales disponibles de la facultad.

\end{itemize}

Para cada entidad antes mencionada existe una tabla en la base de datos del sistema. Algunas de las tablas son: \emph{faculties, careers, years, groups, subjects, semesters, professors, locals}. Por ejemplo, toda carrera pertenece a una facultad, cada grupo pertenece a un año, que a su vez pertenece a una carrera y cada asignatura pertenece a un semestre y a un año.

La tabla \emph{partitions} se utiliza para conocer dada una semana la cantidad y el tipo de turnos que se deben impartir de una asignatura. Generalmente en cada semana se debe impartir una conferencia y una clase pr\'actica de cada asignatura perteneciente al semestre, pero esto no siempre sucede as\'i, existen asignaturas que cambian su planificaci\'on en dependencia si la semana es par o impar. Por otro lado, la tabla \emph{distributions} dado un grupo y una asignatura especifica el profesor que le debe impartir dicha asignatura al grupo dado.

\subsection{Funci\'on de felicidad del sistema}

El Sistema de Gesti\'on de Horarios tiene un m\'odulo dedicado a la especificaci\'on de restricciones. En \'el cada profesor puede especificar sus condiciones o preferencias. Una gran variedad de restricciones pueden ser modeladas, por ejemplo, que un profesor prefiera que todos sus turnos de clases sean en un aula determinada o en un d\'ia espec\'ifico de la semana, que los turnos de conferencia sean en los primeros d\'ias de la semana o que dos turnos de conferencia de una asignatura sean en turnos consecutivos.

Cada profesor puede definir varias restricciones y a cada una de ellas le puede asociar una prioridad. Pueden existir restricciones a las que el profesor le gustar\'ia que se tomaran en cuenta y pueden existir otras restricciones que tengan una importancia de primer orden para \'el, por ejemplo la imposibilidad de impartir clases un determinado d\'ia de la semana, a estas \'ultimas restricciones el profesor debe darles una prioridad alta.

El sistema define la felicidad de un profesor como el grado de cumplimiento de sus restricciones en el horario actual. Dicha medida es calculada por la siguiente funci\'on:

\[ f_{p} = \frac{\sum\limits_{i=1}^n w_{ip} c_{ip}}{\sum\limits_{i=1}^n w_{ip}} \]

\begin{description}

\item[$w_{ip} \in \{1 \ldots 100\}$]: prioridad de la restricci\'on \emph{i} del profesor \emph{p}
\item[$c_{ip} \in \lbrack 0,1 \rbrack$]: grado de cumplimiento de la restricci\'on \emph{i} del profesor \emph{p}
\item[$n$]: cantidad de restricciones definidas por el profesor \emph{p}
\item[$f_{p} \in \lbrack 0,1 \rbrack$]: felicidad del profesor \emph{p}

\end{description}

Tambi\'en se define la felicidad del sistema o el grado de cumplimiento de todas las restricciones impuestas por los profesores. Cada profesor tiene asociada una prioridad en dependencia de su categor\'ia cient\'ifica o docente, adem\'as se tiene en cuenta el cargo que ocupa dentro de la facultad. La felicidad del sistema se calcula de la siguiente forma:

\[ F = \frac{\sum\limits_{i=1}^m f_{i} p_{i}}{\sum\limits_{i=1}^m p_{i}} \]

\begin{description}

\item[$f_{i} \in \lbrack 0,1 \rbrack$]: felicidad del profesor \emph{i}
\item[$p_{i} \in \{1 \ldots 100\}$]: prioridad del profesor \emph{i}
\item[$m$]: cantidad de profesores
\item[$F \in \lbrack 0,1 \rbrack$]: felicidad general del sistema

\end{description}

Debe aclararse que no es el sistema quien decide la forma en la que se toman las decisiones al respecto de los horarios, ni si éstos son confeccionados democrática o dictatorialmente. Esto se decide cuando se asignan las prioridades de los profesores, proceso que se sale de los objetivos de este trabajo. Si se quiere un mecanismo extremadamente democrático, basta con dar a todos la misma prioridad. Un sistema extremadamente dictatorial se lograría dando la misma prioridad a todos los usuarios salvo a uno de ellos, al que se le asignaría una prioridad superior al resto.

\subsection{API de interacci\'on con el Sistema de Gesti\'on de Horarios}

Paralelo a este trabajo se est\'a implementando un API para interactuar con el Sistema de Gesti\'on de Horarios. A trav\'es del API se pueden obtener del sistema todas las entidades necesarias para el diseño del horario. Adem\'as brinda la opci\'on de pasarle un horario en forma de lista de actividades en formato JSON y retornar el grado de felicidad de dicho horario.

\section{Modelaci\'on como un CSP}

Al principio del cap\'itulo ``Problemas de Satisfacción de Restricciones'' se exponen una serie de beneficios que aporta modelar un problema como un CSP. A diferencia de las meta heur\'isticas con que se suelen atacar este tipo de problemas, a los CSPs se le pueden aplicar algoritmos completos, esto quiere decir, que si existe un horario que logre satisfacer todas las restricciones impuestas por los profesores, estos algoritmos completos de seguro la van a encontrar, no sucediendo as\'i con m\'etodos de b\'usqueda local.

En esta secci\'on se explicar\'a la modelaci\'on del problema como un CSP. Se definir\'an las variables con sus valores posibles, las restricciones y c\'omo queda la estructura del grafo de restricciones.

\subsection{Variables y sus dominios de valores}

En el sistema un horario est\'a definido por un conjunto de actividades (tabla \emph{activities}). Cada actividad est\'a compuesta por:

\begin{description}
	\item[Grupo] que va a recibir la clase.
	\item[Asignatura] que se va a tratar.
	\item[Profesor] que va a impartir la clase.
	\item[D\'ia] en que está planificada la clase.
	\item[Turno] del d\'ia en que está planificada la clase.
	\item[Local] donde se va a efectuar la clase.
	\item[Tipo de turno] si es conferencia, clase pr\'actica, laboratorio, etc.
\end{description}

Las variables son tuplas de la forma <grupo, asignatura, tipo de turno> y los valores <día, profesor, turno, local>. Al final cada variable tendr\'a asignado un valor de su dominio, habiendo hecho esto, cada asignaci\'on tendr\'a toda la informaci\'on necesaria para definir una actividad.

Las variables son todas las combinaciones v\'alidas de grupo, asignatura y tipo de turno. Por v\'alida se refiere a que no tiene sentido una variable donde el grupo no reciba dicha asignatura. Igual no tiene sentido incluir la dimensi\'on local dentro de las variables, porque no todos los locales tienen por que estar ocupados, sucede lo mismo con las dimensiones d\'ia, profesor y turno. Sin embargo, el horario tiene que garantizar que todos los grupos den todos los turnos de sus asignaturas correspondientes. Es por eso que se definieron las variables y los valores de esa forma y no de otra. 

\subsection{Restricciones}

Anteriormente se hab\'ia comentado que el sistema nos ofrece una funci\'on de evaluaci\'on de la calidad del horario entre 0 y 1. Sin embargo, los CSPs no necesitan un grado de cumplimiento de las restricciones, solo necesitan saber si las restricciones se cumplen en su totalidad o no. Una posible soluci\'on ser\'ia discretizar el intervalo, o sea, fijar un valor $k$ entre 0 y 1, donde horarios con calidad inferior a $k$ no son aceptados.

Originalmente dicha funci\'on de evaluaci\'on se encontraba dentro del propio sistema, acceder a ella era mediante peticiones HTTP y el proceso de evaluaci\'on hac\'ia un uso excesivo de la base de datos. Todos estos factores hac\'ian el proceso muy lento. En situaciones donde se requiere sucesivas evaluaciones esta propuesta no era factible. Actualmente se est\'a trabajando en exportar la implementaci\'on de la funci\'on hacia un contexto donde no se necesiten hacer peticiones HTTP ni est\'e involucrada la base de datos.

Incluso si se lograra exportar la funci\'on de evaluaci\'on, el c\'alculo de esta es muy complejo y no alcanzar\'ia el nivel de rapidez necesario. Con el objetivo de optimizar el proceso y de disminuir el n\'umero de evaluaciones de la funci\'on de felicidad, se definieron 4 restricciones ajenas al sistema que todo horario debe cumplir:

\begin{description}

\item[Restricci\'on de local]: En un local no se debe impartir m\'as de una clase al mismo tiempo.
\item[Restricci\'on de profesor]: Un profesor no debe tener m\'as de una clase planificada al mismo tiempo.
\item[Restricci\'on de grupo]: Un grupo no debe tener m\'as de una clase asignada al mismo tiempo.
\item[Restricci\'on de orden cronol\'ogico entre actividades]: Se debe respetar el orden cronol\'ogico entre las actividades establecido en la tabla \emph{partitions}.

\end{description}

De esta forma solo se pasa a evaluar la funci\'on de felicidad si el horario cumple las cuatro restricciones anteriores.

\subsection{Estructura del grafo de restricciones}

Todo CSP puede ser representado por un grafo de restricciones, donde los v\'ertices son las variables y las aristas indican que dos variables est\'an relacionadas en alguna restricci\'on. Este grafo es usado en las t\'ecnicas \emph{forward checking} y \emph{arc consistency} donde luego de asignar una variable se analiza las consecuencias que esta pudiera tener en las variables vecinas en el grafo de restricciones.

Analizar la estructura del grafo de restricciones ayudar\'ia a descomponer el problema en subproblemas independientes reduciendo exponencialmente la complejidad del mismo. Si en el grafo existe m\'as de una componente conexa entonces el problema puede ser divido en subproblemas independientes. Las t\'ecnicas \emph{forward checking} y \emph{arc consistency} son favorecidas si el grafo es poco denso.

Desafortunadamente el grafo de restricciones para este problema es completo. En primer lugar, dada la amplia gama de restricciones que pueden ser expresadas en el sistema, obliga a que si una sola actividad es modificada es necesario evaluar todo el horario para obtener su grado de felicidad. En segundo lugar, est\'an las cuatro restricciones definidas anteriormente, por ejemplo, la restricci\'on de local, expresa que no pueden existir dos actividades que requieran del mismo local simult\'aneamente, para ello es necesario chequear todas las variables existentes. 

\section{Algoritmos y t\'ecnicas implementadas}

En el cap\'itulo ``Problemas de Satisfacci\'on de Restricciones'' se menciona que el tamaño del espacio de b\'usqueda para un CSP es $O(d^{n})$, donde $d$ es el tamaño m\'aximo de los dominios de valores y $n$ la cantidad de variables. Aproximadamente existen unas 260 variables por semana, cada semestre tiene unas 17 semanas, dando un total de $4\,420$ variables por semestre. El tamaño promedio de los dominios de valores es de 180, dando unas $180^{4\,420}$ posibles soluciones, este n\'umero es extremadamente grande.

Con el objetivo de disminuir el n\'umero de variables se asume que las semanas del semestre son independientes una de otras, quiere esto decir que se puede generar el horario por semanas. Aunque el sistema permite modelar restricciones que involucren actividades en semanas diferentes, estas son muy espec\'ificas y se puede prescindir de ellas en aras de disminuir exponencialmente el espacio de b\'usqueda. Asumiendo esto, quedar\'ian unas $180^{260}$ posibles soluciones para el horario de una semana.

La metodolog\'ia que se propone es generar el horario de la primera semana con cualquiera de los algoritmos presentados m\'as adelante. Dado que los horarios de cada semana generalmente son bastantes parecidos, es buena idea generar los restantes horarios a partir del horario de la semana anterior. Para ello es aconsejable utilizar un algoritmo de b\'usqueda local usando la heur\'istica \emph{Min-Conflicts} o el algoritmo b\'asico de b\'usqueda \emph{Backtracking Recursive}, para este \'ultimo por cada dominio de variable se pone como primer valor el escogido en el horario anterior, de esta forma se trata de asignar primero los valores que tuvieron \'exito en la semana anterior logrando que los horarios entre semanas se parezcan. Puede que las t\'ecnicas de filtrado \emph{forward checking} y \emph{arc consistency} sean innecesarias en las restantes semanas, dado que se perder\'ia tiempo filtrando los dominios, si al fin y al cabo es probable que la soluci\'on est\'e al principio del \'arbol de b\'usqueda.

A continuaci\'on se mostrar\'an los algoritmos y t\'ecnicas utilizadas para darle soluci\'on al problema, as\'i como el diseño de la jerarqu\'ia de clases.

\subsection{Backtracking Recursive}

La clase \textsf{RecursiveBacktracking} es la base de la jerarqu\'ia, sobre ella se aplican todas las heur\'isticas y t\'ecnicas de filtrado. Su creaci\'on requiere de los par\'ametros \textsf{domains} y \textsf{constraints}. El par\'ametro \textsf{domains} es un diccionario donde las llaves son las variables y los valores son arrays indicando el dominio de valores de cada variable y el par\'ametro \textsf{constraints} es un array de m\'etodos que devuelven \emph{true} o \emph{false} en dependencia si la restricci\'on se cumple. En estos dos par\'ametros se encuentra toda la informaci\'on necesaria para definir un CSP.

En la figura \ref{rb_ruby} se muestra el m\'etodo principal \textsf{recursive\_backtracking} de la clase \textsf{RecursiveBacktracking}. El par\'ametro \textsf{assignment} es la asignaci\'on que se va construyendo en cada llamado recursivo. Esta es una implementaci\'on en Ruby 1.9 del m\'etodo expuesto en el cap\'itulo ``Problemas de Satisfacci\'on de Restricciones''. Los nombres de los m\'etodos utilizados en \'el dan una clara idea de lo que hacen.

\begin{figure}
	\begin{center}
		\includegraphics[scale=0.95]{Graphics/rb_ruby}
		\caption{M\'etodo \textsf{recursive\_backtracking} de la clase \textsf{RecursiveBacktracking}.}
		\label{rb_ruby}
	\end{center}	
\end{figure}

\subsection{Minimum Remaining Values}

En la figura \ref{mrv_ruby} se muestra la clase \textsf{MinimumRemainingValues} que hereda de \textsf{RecursiveBacktracking}. Esta clase lo \'unico que hace es sobrescribir el m\'etodo \textsf{select\_unassigned\_variable}, retornando la variable sin asignar de menor dominio. Este orden es est\'atico, o sea, no cambia durante la b\'usqueda, a menos que se apliquen las t\'ecnicas de filtrado \emph{forward checking} y \emph{arc consistency} que producen un cambio en los dominios de valores durante la b\'usqueda.

\begin{figure}[h]
	\begin{center}
		\includegraphics[scale=0.88]{Graphics/mrv_ruby}
		\caption{La clase \textsf{MinimumRemainingValues}.}
		\label{mrv_ruby}
	\end{center}	
\end{figure}

\subsection{Forward Checking}

La clase \textsf{ForwardChecking} hereda de \textsf{MinimumRemainingValues} solamente para aplicar la heur\'istica \emph{Minimum Remaining Values}. En la clase \textsf{ForwardChecking} se tuvo que redefinir el m\'etodo \textsf{recursive\_backtracking} (ver figura \ref{fc_1_ruby}) por tres razones:

\begin{itemize}

\item Luego de cada asignaci\'on se tienen que filtrar los dominios de las variables vecinas, para ello se llama al m\'etodo \textsf{filter} el cual es un alias al m\'etodo \textsf{forward\_checking}.
\item Dado que los valores en cada dominio son consistentes con las asignaciones previas, producto de haber aplicado \textsf{forward\_checking} tras cada asignaci\'on, no es necesario chequear consistencia antes de asignar un valor a una variable.
\item Cuando se retorna de un llamado recursivo los dominios de cada variable deben quedar como estaban antes de hacer el llamado, para ello en la variable \textsf{@domains\_stack} se guarda el estado de los dominios antes de hacer un llamado recursivo.

\end{itemize}

\begin{figure}[h]
	\begin{center}
		\includegraphics[scale=0.95]{Graphics/fc_1_ruby}
		\caption{M\'etodo \textsf{recursive\_backtracking} de la clase \textsf{ForwardChecking}.}
		\label{fc_1_ruby}
	\end{center}	
\end{figure}

En el m\'etodo \textsf{forward\_checking} de la clase \textsf{ForwardChecking} (ver figura \ref{fc_2_ruby}) se chequean todas las variables vecinas a la variable asignada y los valores que no sean consistente con la asignaci\'on son eliminados de dichos dominios.

\begin{figure}[h]
	\begin{center}
		\includegraphics[scale=0.9]{Graphics/fc_2_ruby}
		\caption{M\'etodo \textsf{forward\_checking} de la clase \textsf{ForwardChecking}.}
		\label{fc_2_ruby}
	\end{center}	
\end{figure}

\subsection{Arc Consistency}

La clase \textsf{ArcConsistency} hereda de \textsf{ForwardChecking}, dado que la t\'ecnica \emph{arc consistency} es una extensi\'on de \emph{forward checking}, donde luego de eliminar valores inconsistentes se vuelven a chequear los arcos que apuntan a la variable con dichos valores. Con esto se logra reutilizar el m\'etodo \textsf{recursive\_backtracking} redefinido en la clase \textsf{ForwardChecking}.

\begin{figure}[h]
	\begin{center}
		\includegraphics[scale=0.75]{Graphics/AC3_1_ruby}
		\caption{M\'etodo \textsf{AC3} de la clase \textsf{ArcConsistency}.}
		\label{AC3_1_ruby}
	\end{center}	
\end{figure}

\begin{figure}[h]
	\begin{center}
		\includegraphics[scale=0.67]{Graphics/AC3_2_ruby}
		\caption{M\'etodo \textsf{remove\_inconsistent\_values} de la clase \textsf{ArcConsistency}.}
		\label{AC3_2_ruby}
	\end{center}	
\end{figure}

Las figuras \ref{AC3_1_ruby} y \ref{AC3_2_ruby} muestran una implementaci\'on de los m\'etodos para consistencia de arcos descritos en el cap\'itulo ``Problemas de Satisfacci\'on de Restricciones''. El m\'etodo \textsf{AC3} es un alias al m\'etodo \textsf{filter} llamado en \textsf{recursive\_backtracking} de la clase \textsf{ForwardChecking}.

\subsection{B\'usqueda Local con Min-Conflicts}

En la figura \ref{mc_ruby} se muestra una implementaci\'on del algoritmo de b\'usqueda local con \emph{min-conflicts} abordado en la secci\'on ``B\'usqueda Local para Problemas de Satisfacci\'on de Restricciones'' del cap\'itulo ``Problemas de Satisfacci\'on de Restricciones''. 

\begin{figure}[h]
	\begin{center}
		\includegraphics[scale=0.7]{Graphics/mc_ruby}
		\caption{M\'etodo \textsf{min\_conflicts} de la clase \textsf{MinConflicts}.}
		\label{mc_ruby}
	\end{center}	
\end{figure}

En el algoritmo, de las variables inconsistentes se selecciona una al azar:
\begin{center}	\textsf{var = inconsistent\_variables.sample}	\end{center}

Luego de esa variable se selecciona su mejor valor, es decir, el valor que cumpla la mayor cantidad de restricciones posibles:
\begin{center}	\textsf{value = best\_value(var, assignment)}	\end{center}

