\chapter{Propuesta de Solución}

En este cap\'itulo se presenta una posible soluci\'on al problema de generar un horario universitario de clases de forma autom\'atica, para ello se modelar\'a como un Problema de Satisfacci\'on de Restricciones. Del Sistema de Gesti\'on de Horarios se explicar\'a la modelaci\'on de la entidades involucradas, la funci\'on de felicidad del sistema y un API para interactuar con el mismo. Luego se mostrar\'a c\'omo se definieron las variables y los dominios de valores de las mismas, las restricciones del problema y la estructura del grafo de restricciones. Por \'ultimo se examinar\'an los algoritmos y t\'ecnicas utilizadas para darle soluci\'on al problema.

\section{Interacci\'on con el Sistema de Gesti\'on de Horarios}

 Horarios se explicar\'a la modelaci\'on de la entidades involucradas, la funci\'on de felicidad del sistema y un API para interactuar con el mismo. Luego se mostrar\'a c\'omo se definieron las variables y los dominios de valores de las m

\subsection{Modelaci\'on de las entidades involucradas}

\subsection{Funci\'on de felicidad del sistema}

\subsection{API de interacci\'on con el Sistema de Gesti\'on de Horarios}

\section{Modelaci\'on como un CSP}

\subsection{Variables y sus dominios de valores}

\subsection{Restricciones}

\subsection{Estructura del grafo de restricciones}

\section{Algoritmos y t\'ecnicas implementadas}

\subsection{Backtracking Recursive}

\subsection{Minimum Remaining Values}

\subsection{Forward Checking}

\subsection{Arc Consistency}

\subsection{B\'usqueda Local con Min-Conflicts}