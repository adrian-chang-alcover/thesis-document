\chapter{Propuesta de Solución}

En este cap\'itulo se presenta una posible soluci\'on al problema de generar un horario universitario de clases de forma autom\'atica, para ello se modelar\'a como un Problema de Satisfacci\'on de Restricciones. Del Sistema de Gesti\'on de Horarios se explicar\'a la modelaci\'on de la entidades involucradas, la funci\'on de felicidad del sistema y un API para interactuar con el mismo. Luego se mostrar\'a c\'omo se definieron las variables y los dominios de valores de las mismas, las restricciones del problema y la estructura del grafo de restricciones. Por \'ultimo se examinar\'an los algoritmos y t\'ecnicas utilizadas para darle soluci\'on al problema.

\section{Interacci\'on con el Sistema de Gesti\'on de Horarios}

Como ya se hab\'ia mencionado este trabajo formar\'a parte de un Sistema de Gesti\'on de Horarios desarrollado por el profesor Lic. Pedro Quintero Rojas perteneciente a la facultad de Matem\'atica y Computaci\'on de la Universidad de La Habana. Dicho sistema consiste en un sitio web implementado en Ruby on Rails. La versi\'on del int\'erprete de Ruby utilizado fue 1.9 y la versi\'on de Rails fue 3.2.13, para la base de datos se utiliz\'o SQLite3.

El sistema contiene toda la informaci\'on necesaria para generar un horario de clases. Tambi\'en muestra una vista actualizada del horario vigente y las personas autorizadas pueden modificar turnos planificados y reservar aulas de manera transparente. En un futuro se desea implementar una herramienta para tel\'efonos celulares donde se pueda interactuar con el sistema y de alguna forma avisar a las personas involucradas cuando se cambie un turno de clase. 

\subsection{Modelaci\'on de las entidades involucradas}

Para generar un horario de clases de un semestre para una facultad es necesario tener de antemano toda la informaci\'on referente. Se necesita conocer:

\begin{itemize}

\item Las carreras que se estudian en la facultad.
\item Los años docentes que tiene cada carrera.
\item Cantidad de grupos por año docente.
\item Por cada año docente conocer las asignaturas que se deben dar en el semestre en cuesti\'on.
\item Por cada semana del semestre conocer la cantidad y el tipo de turno que se deben impartir de cada asignatura.
\item Los profesores que imparten cada asignatura.
\item Capacidad y tipo de locales disponibles de la facultad.

\end{itemize}

Para cada entidad antes mencionada existe una tabla en la base de datos del sistema. Algunas de las tablas son: \emph{faculties, careers, years, groups, subjects, semesters, professors, locals}. Por ejemplo toda carrera pertenece a una facultad, cada grupo pertenece a un año que a su vez pertenece a una carrera y cada asignatura pertenece a un semestre y a un año.

La tabla \emph{partitions} se utiliza para conocer dado una semana la cantidad y el tipo de turnos que se deben impartir de una asignatura. Generalmente en cada semana se debe impartir una conferencia y una clase pr\'actica de cada asignatura perteneciente al semestre, pero esto no siempre sucede as\'i, existen asignaturas que cambian su planificaci\'on en dependencia si la semana es par o impar. Por otro lado, la tabla \emph{distributions} dado un grupo y una asignatura especifica el profesor que le debe impartir dicha asignatura al grupo dado.

\subsection{Funci\'on de felicidad del sistema}

El Sistema de Gesti\'on de Horarios tiene un m\'odulo dedicado a la especificaci\'on de restricciones. En \'el cada profesor puede especificar sus condiciones o preferencias. Una gran variedad de restricciones pueden ser modeladas, por ejemplo, que un profesor prefiera que todos sus turnos de clases sean en un aula determinada o en un d\'ia espec\'ifico de la semana, que los turnos de conferencia sean en los primeros dias de la semana o que dos turnos de conferencia de una asignatura sean en turnos consecutivos.

Cada profesor puede definir varias restricciones y a cada una de ellas le puede asociar una prioridad. Pueden existir restricciones a las que el profesor le gustar\'ia que se tomaran en cuenta y pueden exitir otras restricciones que tengan una importacia de primer orden para \'el, por ejemplo la imposibilidad de impartir clases un determinado d\'ia de la semana, a estas \'ultimas restricciones el profesor debe darles una prioridad alta.

El sistema define la felicidad de un profesor como el grado de cumplimiento de sus restricciones en el horario actual. Dicha medida es calculada por la siguiente funci\'on:

\[ f_{p} = \frac{\sum\limits_{i=1}^n w_{ip} c_{ip}}{\sum\limits_{i=1}^n w_{ip}} \]

\begin{description}

\item[$w_{ip} \in \{1 \ldots 100\}$]: prioridad de la restricci\'on \emph{i} del profesor \emph{p}
\item[$c_{ip} \in \lbrack 0,1 \rbrack$]: grado de cumplimiento de la restricci\'on \emph{i} del profesor \emph{p}
\item[$n$]: cantidad de restricciones definidas por el profesor \emph{p}
\item[$f_{p} \in \lbrack 0,1 \rbrack$]: felicidad del profesor \emph{p}

\end{description}

Tambi\'en se define la felicidad del sistema o el grado de cumplimiento de todas las restricciones impuestas por los profesores. Cada profesor tiene asociada una prioridad en dependencia de su categor\'ia cient\'ifica o docente, adem\'as se tiene en cuenta el cargo que ocupa dentro de la facultad. La felicidad del sistema se calcula de la siguiente forma:

\[ F = \frac{\sum\limits_{i=1}^m f_{i} p_{i}}{\sum\limits_{i=1}^m p_{i}} \]

\begin{description}

\item[$f_{i} \in \lbrack 0,1 \rbrack$]: felicidad del profesor \emph{i}
\item[$p_{i} \in \{1 \ldots 100\}$]: prioridad del profesor \emph{i}
\item[$m$]: cantidad de profesores
\item[$F \in \lbrack 0,1 \rbrack$]: felicidad general del sistema

\end{description}

Debe aclararse que no es el sistema quien decide la forma en la que se toman las decisiones al respecto de los horarios, ni si éstos son confeccionados democrática o dictatorialmente. Esto se decide cuando se asignan las prioridades de los profesores, proceso que se sale de los objetivos de este trabajo. Si se quiere un mecanismo extremadamente democrático, basta con dar a todos la misma prioridad. Un sistema extremadamente dictatorial se lograría dando la misma prioridad a todos los usuarios salvo a uno de ellos, al que se le asignaría una prioridad superior al resto.

\subsection{API de interacci\'on con el Sistema de Gesti\'on de Horarios}

esperando por Gabriel...

\section{Modelaci\'on como un CSP}

Al principio del cap\'itulo ``Problemas de Satisfacción de Restricciones'' se exponen una serie de beneficios que aporta modelar un problema como un CSP. A diferencia de las meta heur\'isticas con que se suelen atacar este tipo de problemas, a los CSPs se le pueden aplicar algoritmos completos, esto quiere decir, que si existe un horario que logre satisfacer todas las restricciones impuestas por los profesores, estos algoritmos completos de seguro la van a encontrar, no sucediendo as\'i con m\'etodos de b\'uqueda local.

En esta secci\'on se explicar\'a la modelaci\'on del problema como un CSP. Se definir\'an las variables con sus valores posibles, las restricciones y c\'omo queda la estructura del grafo de restricciones.

\subsection{Variables y sus dominios de valores}

En el sistema un horario est\'a definido por un conjunto de actividades (tabla \emph{activities}). Cada actividad est\'a compuesta por:

\begin{description}
	\item[Grupo] que va a recibir la clase.
	\item[Asignatura] que se va a tratar.
	\item[Profesor] que va a impartir la clase.
	\item[D\'ia] en que está planificada la clase.
	\item[Turno] del d\'ia en que está planificada la clase.
	\item[Local] donde se va a efectuar la clase.
	\item[Tipo de turno] si es conferencia, clase pr\'actica, laboratorio, etc.
\end{description}

Las variables son tuplas de la forma <grupo, asignatura, tipo de turno> y los valores <día, profesor, turno, local>. Al final cada variable tendr\'a asignado un valor de su dominio, habiendo hecho esto, cada asignaci\'on tendr\'a toda la informaci\'on necesaria para definir una actividad.

Las variables son todas las combinaciones v\'alidas de grupo, asignatura y tipo de turno. Por v\'alida se refiere a que no tiene sentido una variable donde el grupo no reciba dicha asignatura. Igual no tiene sentido incluir la dimensi\'on local dentro de las variables, porque no todos los locales tienen por que estar ocupados, sucede lo mismo con las dimensiones d\'ia, profesor y turno. Sin embargo, el horario tiene que garantizar que todos los grupos den todos los turnos de sus asignaturas correspondientes. Es por eso que se definieron las variables y los valores de esa forma y no de otra. 

\subsection{Restricciones}

Anteriormente se hab\'ia comentado que el sistema nos ofrece una funci\'on de evaluaci\'on de la calidad del horario entre 0 y 1. Sin embargo, los CSPs no necesitan un grado de cumplimiento de las restricciones, solo necesitan saber si las restricciones se cumplen en su totalidad o no. Una posible soluci\'on ser\'ia discretizar el intervalo, o sea, fijar un valor $k$ entre 0 y 1, donde horarios con calidad inferior a $k$ no son aceptados.

Originalmente dicha funci\'on de evaluaci\'on se encontraba dentro del propio sistema, acceder a ella era mediante peticiones HTTP y el proceso de evaluaci\'on hac\'ia un uso excesivo de la base de datos. Todos estos factores hac\'ian el proceso muy lento. En situaciones donde se requiere sucesivas evaluaciones esta propuesta no era factible. Actualmente se est\'a trabajando en exportar la implementaci\'on de la funci\'on hacia un contexto donde no se necesiten hacer peticiones HTTP ni est\'e involucrada la base de datos.

Incluso si se lograra exportar la funci\'on de evaluaci\'on, el c\'alculo de esta es muy complejo y no alcanzar\'ia el nivel de rapidez necesario. Con el objetivo de optimizar el proceso y de disminuir el n\'umero de evaluaciones de la funci\'on de felicidad, se definieron 4 restricciones ajenas al sistema que todo horario debe cumplir:

\begin{description}

\item[Restricci\'on de local]: En un local no se debe impartir m\'as de una clase al mismo tiempo.
\item[Restricci\'on de profesor]: Un profesor no debe tener m\'as de una clase planificada al mismo tiempo.
\item[Restricci\'on de grupo]: Un grupo no debe tener m\'as de una clase asignada al mismo tiempo.
\item[Restricci\'on de orden cronol\'ogico entre actividades]: Se debe respetar el orden cronol\'ogico entre las actividades entablecido en la tabla \emph{partitions}.

\end{description}

De esta forma solo se pasa a evaluar la funci\'on de felicidad si el horario cumple las cuatro restricciones anteriores.

\subsection{Estructura del grafo de restricciones}

Todo CSP puede ser representado por un grafo de restricciones, donde los v\'ertices son las variables y las aristas indican que dos variables est\'an relacionadas en alguna restricci\'on. Este grafo es usado en las t\'ecnicas \emph{forward checking} y \emph{arc consistency} donde luego de asignar una variable se analiza las consecuencias que esta pudiera tener en las variables vecinas en el grafo de restricciones.

Analizar la estructura del grafo de restricciones ayudar\'ia a descomponer el problema en subproblemas independientes reduciendo exponencialmente la complejidad del mismo. Si en el grafo existen m\'as de una componente conexa entonces el problema puede ser divido en subproblemas independientes. Las t\'ecnicas \emph{forward checking} y \emph{arc consistency} son favorecidas si el grafo es poco denso.

Desafortunadamente el grafo de restricciones para este problema es completo.

\section{Algoritmos y t\'ecnicas implementadas}

\subsection{Backtracking Recursive}

\subsection{Minimum Remaining Values}

\subsection{Forward Checking}

\subsection{Arc Consistency}

\subsection{B\'usqueda Local con Min-Conflicts}