\chapter{An\'alisis de Resultados}

En este cap\'itulo se analizan los resultados de haber aplicado los algoritmos implementados a diferentes CSPs. Para ello probamos dos instancias del problema de los horarios obtenidos mediante la modificaci\'on y la adici\'on de restricciones. Se comparan distintas metodolog\'ias para generar el horario completo de un semestre. Adem\'as se implement\'o el problema de las n-reinas y un problema de criptograf\'ia sencillo con el objetivo de comparar los algoritmos implementados frente a diferentes problemas.

\section{Comparaci\'on entre distintas instancias del problema de los horarios}

En esta secci\'on se analizan los resultados de generar el horario de una semana para dos instancias diferentes del problema de los horarios utlizando los algoritmos implementados. La primera instancia probada, ``instancia f\'acil'', consta de solo las cuatros restricciones vistas en la secci\'on ``Modelaci\'on como un CSP'' del cap\'itulo ``Propuesta de Solución'', las restricciones de local, de profesor, de grupo y de orden cronol\'ogico entre actividades.

\begin{table}[h]
	\caption{Resultado de haber aplicado los algoritmos implementados a distintas instancias del problema de los horarios}
	\begin{center}
		\label{1semana}
		\begin{tabular}{l|c|c}
			Algoritmos & Instancia f\'acil & Instancia dura \\ \hline
			RB & 15 seg, 326 nodos & 8 h y 35 min, $229\,210$ nodos\\
			MRV & > 10 h, $316\,207$ nodos &  > 10 h, $242\,960$ nodos\\
			FC & 718 seg, 441 nodos & 1 h y 56 min, $2\,469$ nodos \\
			AC & > 10 h, 133 nodos &  > 10 h, 98 nodos \\
			MC & 159 seg, 205 iteraciones & \'optimo local, 704 iteraciones
		\end{tabular}
	\end{center}
\end{table}

\section{Comparaci\'on entre distintas metodolog\'ias para generar un horario completo}

\section{Comparaci\'on entre distintos problemas}