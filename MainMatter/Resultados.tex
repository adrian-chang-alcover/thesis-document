\chapter{An\'alisis de Resultados}

En este cap\'itulo se analizan los resultados de haber aplicado los algoritmos implementados a diferentes CSPs. Para ello probamos dos instancias del problema de los horarios obtenidos mediante la modificaci\'on y la adici\'on de restricciones. Se comparan distintas metodolog\'ias para generar el horario completo de un semestre. Adem\'as se implement\'o el problema de las n-reinas y un problema de criptograf\'ia sencillo con el objetivo de comparar los algoritmos implementados frente a diferentes problemas.

\section{Comparaci\'on entre distintas instancias del problema de los horarios}

En esta secci\'on se analizan los resultados de generar el horario de una semana para dos instancias diferentes del problema de los horarios utilizando los algoritmos implementados. La primera instancia probada, ``instancia f\'acil'', consta de solo las cuatro restricciones vistas en la secci\'on ``Modelaci\'on como un CSP'' del cap\'itulo ``Propuesta de Solución'', las restricciones de local, de profesor, de grupo y de orden cronol\'ogico entre actividades. A la segunda instancia, ``instancia dura'', se le adicionaron restricciones de preferencias de profesores con respecto a d\'ias de la semana o secci\'on del d\'ia. Adem\'as se le adicion\'o la restricci\'on que toda actividad fuera de la facultad debe ser dada al mismo tiempo por todos los grupos del año. Ambas instancias del problema tienen 274 variables y el tamaño promedio de los dominios es 180.

\begin{table}[h]
	\caption{Resultado de haber aplicado los algoritmos implementados a distintas instancias del problema de los horarios}
	\begin{center}
		\label{1semana}
		\begin{tabular}{|l|c|c||}
			\hline \hline
			Algoritmos & Instancia f\'acil & Instancia dura \\ \hline
			RB & 15 segundos & 8 horas y 35 minutos \\
			 & 326 nodos explorados & $229\,210$ nodos explorados \\ \hline
			MRV & m\'as de 10 horas &  m\'as 10 horas \\
			 & $316\,207$ nodos explorados & $242\,960$ nodos explorados \\ \hline
			FC & 718 segundos & 1 hora y 56 minutos \\
			 & 441 nodos explorados & $2\,469$ nodos explorados \\ \hline
			AC & m\'as de 10 horas &  m\'as de 10 horas \\
			 & 133 nodos explorados & 98 nodos explorados \\ \hline
			MC & 159 segundos & \'optimo local \\
			 & 205 iteraciones & 704 iteraciones \\ \hline
		\end{tabular}
	\end{center}
\end{table}

En la tabla \ref{1semana} se muestran los resultados. El algoritmo que m\'as r\'apido resuelve la ``instancia f\'acil'' es el \textsf{RecursiveBacktracking}, esto se debe a que existe una soluci\'on al principio del espacio de b\'usqueda. La prontitud con que se encuentra una soluci\'on depende mucho del orden de las variables, es por eso que la heur\'istica \textsf{MinimumRemainingValues} en 10 horas no encuentra soluci\'on en ambos casos a pesar de la enorme cantidad de nodos explorados.

\section{Comparaci\'on entre distintas metodolog\'ias para generar un horario completo}

\section{Comparaci\'on entre distintos problemas}