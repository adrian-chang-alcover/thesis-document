\chapter{Estado del Arte}

\section{Timetabling}

El diseño de un horario de clases o de exámenes para una universidad es una tarea bastante grande y compleja.
Existen diferentes departamentos y facultades, cada uno con sus propias ideas acerca de cómo y cuándo deben
comenzar sus cursos. Además, los estudiantes pueden tomar cursos de diferentes departamentos e incluso de
diferentes facultades. Numerosos sistemas de horarios para universidades han sido diseñados y se trabaja
en dirección a crear un estándar que permita comparar un sistema con otro. En esta sección se resumirán
las principales técnicas y trabajos recientes en cuanto a horarios, sea de clases o de exámenes, y servirá
como una introducción a este campo de investigación.

\subsection{Definición del problema}

Un horario (timetable en Inglés) es una asignación de un conjunto de encuentros en un tiempo determinado.
Un encuentro es una combinación de recursos (aulas, personas o materiales de clase), algunos de ellos pueden
ser especificados por el problema y otros deben ser asignados como parte de la solución. La generación de un
horiario se conoce que pertenece a la clase de problemas llamada NP-completo \cite{TB Cooper and JH Kingston},
hasta ahora no se conoce un algoritmo polinomial que pueda darle solución.

Existen algunas variaciones fundamentales para el problema de los horarios. La generación de un horario para
la universidad puede ser dividido en dos principales categorias, horario para clases y horario para exámenes.
Se diferencian principalmente en:

\begin{itemize}
	\item Los exámenes deben ser planificados de forma tal que un estudiante no tenga más de un exámen al
		mismo tiempo, pero las clases usualmente son planificadas antes que los estudiantes las matriculen.
	\item El espacio también es una restricción, diferentes exámenes pueden compartir la misma aula o un exámen
		puede necesitar más de un aula, pero en un aula solo puede ser impartida una clase a la vez.
\end{itemize}

Carter defiende con respecto al problema del horario de exámenes: ``\emph{el principal reto es programar los
exámenes para un período de tiempo limitado evitando conflictos y satisfaciendo un número de restricciones
colaterales}'' \cite{Carter's summary} Con conflictos se refiere a que el horario requiera de un recurso
al mismo tiempo en lugares distintos. Las restricciones colaterales varian entre instituciones, y entre
horarios para clases y horarios para exámenes. Pueden existir una infinidad de restricciones colaterales.
Reto es la palabra apropiada para usar en estos casos.
Como Bloomfield y McSharry dicen: ``\emph{Dependiendo del tamaño de los departamentos y de la diversidad
de cursos que ofrecen, el tiempo requerido para diseñar un horario de clases puede ir desde una simple
tarde de trabajo hasta un mes de arduo trabajo}''. \cite{Bloomfield and McSharry says}

El proceso de diseño de un horario se vuelve más difícil por el hecho de que involucra a muchas personas.
Romero indentificó tres grupos fundamentales cada uno con sus propios objetivos y necesidades \cite{Romero}.

\begin{itemize}
	\item La administración fija los estándares  mínimos que el horario debe cumplir. Por ejemplo, algunas
		universidades especifican que los estudiantes no tengan dos exámenes en períodos consecutivos.
	\item Los intereses de cada departamento son los más notables en el diseño de un horario. Cada departamento
		quiere que el horario sea consecuente con las asignaturas que imparte, asi como la necesidad de un
		aula o laboratorio en específico. En el contexto de exámenes, se quisiera que los exámenes más complicados
		sean los primeros, para asi tener más tiempo para su calificación.
	\item El tercer grupo involucrado son los estudiantes, a quienes solo le interesará la porción de horario
		referente a ellos. Dada la cantidad de estudiantes y la diversidad de criterios que estos puedan tener,
		se hace muy difícil determinar cual es el mejor horario para ellos. Muchos estudiantes prefieren no
		tener clases lo viernes y entre cada clases tener un descanso.		
\end{itemize}

Las restricciones para un horario pueden ser muchas y variadas. A continuación se presentan algunas de 
las más comunes:

\begin{description}
	\item[Asignación de recursos:] Un recurso puede ser asignado a otro recurso de tipo diferente o a un encuentro.
		Por ejemplo, un profesor prefiere impartir sus clases en una determinada aula o un exámen debe ser realizado
		en un determinado edificio.
	\item[Asignación de tiempo:] Un encuentro o un recurso debe ser asignado a un tiempo. Esta restricción puede
		ser usada para específicar días en los que un profesor no está disponible, o para pre-asignar un
		tiempo a un determinado encuentro.
	\item[Restricción de tiempo entre encuentros:] Ejemplos comunes de esta clase de restricción son que un encuentro
		deba suceder antes de otro, o que un conjunto de exámenes deba realizarse simultaneamente.
	\item[Esparcimiento de encuentros:] Los encuentros deben estar esparcidos en el tiempo. Por ejemplo, un
		estudiante no debe tener más de un exámen el mismo día.
	\item[Coherencia entre encuentros:] Estas restricciones son diseñadas para producir un horario más organizado
		y conveniente. A menudo entran en contradicción con el esparcimiento de encuentros. Ejemplos de ellas es,
		que un profesor prefiera dar todos sus turnos en los primeros tres dias de la semana, dejándole los
		últimos dos dias libre.
	\item[Capacidad de las aulas:] La cantidad de estudiantes no debe exceder la capacidad del aula.
	\item[Continuidad:] Cualquier restricción cuyo propósito sea asegurar que el horario sea constante y predecible.
		Por ejemplo, las conferencias de una asignatura deben ser programadas en la misma aula o en el mismo tiempo.
\end{description}

Las restricciones son usualmente dividadas entre las categorias fuertes y débiles:

\begin{description}
	\item[Restricciones fuertes:] Un horario que no cumpla con alguna restricción fuerte no es una solución factible
		y debe ser modificado. Ejemplo, una persona no debe tener planificado más de un encuentro al mismo tiempo.
	\item[Restricciones débiles:] Estas restricciones son menos importantes que las fuertes y es casi imposible
		cumplirlas todas. En algunas ocasiones se mide la calidad de un horario por el número de restricciones
		débiles que logra cumplir. Algunas restricciones débiles son más importantes que otras, esto puede ser
		específicado asociandoles una prioridad.
\end{description}

\subsection{Encuesta realizada en universidades británicas}

El problema de los horarios puede variar enormente entre distintas universidades. El Grupo de Planificación
y Programación Automatizada de la Universidad de Nottingham lanzó una encuesta \cite{survey of University of Nottingham} sobre la planificación de exámenes en las universidades británicas. La encuesta fue enviada a
95 universidades, de ellas respondieron 56.

La encuesta mostró cuán distinto puede ser el problema de los horarios entre las diferentes instituciones.
Por ejemplo, el número de exámenes a planificar puede variar desde unos cientos hasta medio millar por sesión,
y los estudiantes involucrados pueden ser desde unos quinientos hasta unos veinte mil. Como dijo un
encargado de programar el horario, ``\emph{Existen muchas variaciones en cuanto a la generación de un horario,
podriamos escribir un libro}''.

Uno de los mayores problemas para las universidades es encontrar locales en los que realizar los exámenes.
Varias universidades resuelven el problema de la insuficiencia de aulas alquilando locales externos a la
universidad. Esto puede resolver el problema en parte, pero si estos locales son utilizados en otras funciones
su disponibilidad se vuelve impredecible. También hay que considerar que estos locales no siempre están cerca de
la universidad, causando que los estudiantes necesiten más tiempo para trasladarse de un local a otro.

De las universidades que respondieron la encuesta, sorprende que el 42\% no se apoyan en ninguna herramienta
computacional para realizar este proceso. Una posible razón para esto es que en varias universidades el
horario de un curso no cambia significativamente con respecto al anterior. Aquellas universidades que no
se apoyan en el horario del curso anterior ni en alguna herramienta computacional demoran al menos cuatro
semanas en diseñar su horario.

Otra razón para la falta de automatización en este proceso es producto de que los encargados de diseñar el horario
algunas veces desconfian de los beneficios que esto podria traer. El departamento de exámenes de una universidad
dijo que ellos podian ``\emph{ver las desventajas de un sistema completamente automatizado. Preferimos trabajar
desde la etapa inicial del diseño con una tabla y un lapiz. Esto permite una vista general del progreso de la
generación mejor que un sistema computacional basado en preguntas.}''

\begin{figure}
	\begin{center}
		\includegraphics{Graphics/computer_usage}
		\caption{Automatización del proceso de diseño de horarios en universidades británicas.}
	\end{center}	
\end{figure}

De las universidades que respondieron la encuesta el 21\% genera su horario de forma automática por
computadoras, aunque confiesan que a veces es necesario la intervención manual con el objetivo de satisfacer
algunas particularidades que escapan del alcance de las computadoras. Algunas universidades desarrollan
sus propias herramientas y otras universidades adaptan las herramientas comerciales a sus necesidades.

Exiten dos metodologías fundamentales que utilizan aquellas universidades que no automatizan el proceso de
generación de horarios:

\begin{itemize}
	\item Permitir a cada facultad o departamento diseñar su propuesta de horario y luego la administración
		central los mezcla y elabora el horario final. Esto asume que las facultades y los departamentos
		son independientes entre ellos. Sin embargo, muchas universidades británicas están adoptando una
		organización modular, lo que implica que esta metodología no es siempre válida.
	\item Construir el horario de exámenes modificando el horario de clases. Una institución que casualmente
		estaba desarrollando un nuevo sistema, dijo: ``\emph{Esto, por supuesto no está del todo bien, porque
		pueden haber varias conferencias en diferentes intervalos de tiempo o el número de estudiantes involucrados
		es mayor a la capacidad de las aulas de exámenes}''. Esta metodología depende de la naturaleza del curso
		asociado al horario.
\end{itemize}

La encuesta demostró la gran necesidad de automatizar este proceso. Habiendo dicho esto, cualquier sistema
para que sea útil debe satisfacer la gran cantidad de criterios disimiles que existen entorno a este problema.
El principal problema es que el sistema debe ser capaz de generar horarios de gran calidad a pesar de las enormes
variaciones de este proceso encontradas en las diferentes universidades encuestadas. El sistema debe ser compatible
con trabajos hechos con anterioridad, debe ser fácil de usar y satisfacer todas las necesidades de los departamentos y facultades.

\subsection{¿Cómo comparar la calidad de las soluciones?}

Varios métodos han sido desarrollados y utilizados con éxito para solucionar el problema de los horarios.
\cite{VA Bardadym, MW Carter, MW Carter and G Laporte, JH Kingston}. Sin embargo, muchos de estos métodos 
solo son usados en un departamento o facultad en específico, rara vez son comparados entre sí. Una 
comparación precisa es vital para determinar cuál de los métodos computacionales es el mejor para
distintas instancias del problema. Cuando la técnica usada es relativamente simple, se ajusta con facilidad
a condiciones distintas. Con el paso del tiempo las técnicas se vuelven más sofisticadas, con el objetivo
de satisfacer nuevas necesidades.

No es fácil expresar claro y precisamente los requerimientos para un horario real. Los requerimientos básicos
como ``cada clase debe tener un profesor'' o ``nadie debe estar en dos lugares distintos al mismo tiempo'' son
fáciles de expresar, pero las ``restricciones colaterales'' como el aula ideal para impartir una asignatura o
el orden cronológico entre clases son muy difíciles de representar y por consiguiente de leer y entender. Esa
es la razón por qué es tan difícil adaptar un sistema hecho para otra institución a nuestras necesidades.

Un proyecto en desarrollo por la Universidad de Nottingham conjunto con la Universidad de Napier, la Universidad
de Reading, la Universidad de Sydney y la Universidad de Toronto intenta encontrar una forma de escribir cualquier
requerimiento, usando fórmulas lógicas. Esto permitirá utilizar herramientas desarrolladas por otras entidades tan
solo redefiniendo nuestros propios requerimientos.

\subsection{Técnicas para automatizar el diseño de un horario}

Para determinadas instancias del problema heurísticas específicas pueden dar buenos resultados, pero con el devenir del tiempo los horarios de universidades se han vuelto más complejos. Actualmente hay una tendencia que fue confirmada
con las presentaciones de la primera conferencia internacional de ``Practice And Theory of Automated Timetabling''
\cite{D Abramson and J Abela}, de resolver el problema con algoritmos más generales, o metaheurísticas, como
recocido simulado, algoritmos evolutivos y búsqueda tabú. Las herísticas específicas pueden ser utilizadas
para disminuir el número de posibles soluciones a procesar o para optimizar localmente una solución. Programación
Lógica de Restricciones (Constraint Logic Programming) es también otra técnica utilizada.

\subsubsection{Algoritmos Genéticos}

Los algoritmos genéticos son análogos a la evolución propuesta por Darwin. Una población de horarios válidos
es mantenida. Los mejores horarios son seleccionados para formar las bases de la próxima generación, mejorando
la calidad de la población al mismo tiempo que se mantiene la diversidad.

La representación genética más común para un horario es una larga cadena de bits codificando cuándo y dónde
debe ocurrir un encuentro. Siendo así, los pares de los horarios seleccionados pueden ser cruzados cortando
la cadena y pasando a sus descendientes información de ambos padres. No obstante, Corne, Ross y Fang proponen
una conveniente operación de mutación para ser más exitoso el cruzamiento \cite{D Corne and P Ross and HL Fang}.
Su sistema, GATT, está siendo usado para generar el horario de la Universidad de Edinburgh y algunas otras
instituciones. La figura \ref{genetic algorithms} muestra las etapas de un algoritmo gen\'etico.

\begin{figure}
	\begin{center}
		\label{genetic algorithms}
		\includegraphics[scale=0.5]{Graphics/genetic_algorithms}
		\caption{Etapas de un algoritmo genético.}		
	\end{center}	
\end{figure}

Paechter y Cumming desarrollaron ``Neeps and Tatties'' un sistema que se ha estado usando en el departamento
de Ciencia de la Computación de la Universidad de Napier. Su algoritmo genético codifica un horario como una ordenación
de eventos, que debe ser introducido a un programa especial que utiliza ese orden para producir un horario
\cite{B Paechter* and A Cumming* and H Luchian}. Este algoritmo necesita un operador de permutación 
que cambie el orden de los eventos heredados por el padre.

El Grupo de Planificación y Programación Automatizada de la Universidad de Nottingham ha estado desarrollando
algoritmos genéticos para horarios de exámenes utilizando un alto grado de conocimiento heurístico para generar
una población inicial y para los operadores genéticos \cite{EK Burke and DG Elliman and RF Weare 1, 
EK Burke and DG Elliman and RF Weare 2, EK Burke and DG Elliman and RF Weare 3}. El operador de cruzamiento
trabaja a nivel de período, tomando encuentros planificados de ambos padres primero, y luego planificando
otros de acuerdo a la heurística de ordenamiento hasta que no se puedan planificar más encuentros. Este tipo de operador de cruzamiento permite un uso eficiente de las aulas porque siempre trata de asignar un encuentro dondequiera que sea posible.

\subsubsection{Algoritmos Mem\'eticos}

Los algoritmos mem\'eticos son una extensión de los algoritmos gen\'eticos, basados en un modelo de cómo las ideas evolucionan. La unidad básica de las ideas son los memes, los que a diferencia de los genes, pueden ser mejorados durante su ciclo de vida. Es por eso que un simple algoritmo de búsqueda local (hill-climbing) es usado por intervalos
para asegurarse que todos los horarios miembros de la población son óptimos locales.

La Universidad de Nottingham se encuentra desarrollando algoritmos mem\'eticos para horarios de exámenes \cite{EK Burke and JP Newall and RF Weare}. Un operador de mutación es aplicado a los miembros seleccionados de la población. Despues de mutar cada miembro seleccionado este es optimizado por un algoritmo de optimización local. Esta combinación genera soluciones con buena calidad.

La Universidad de Napier está trabajando con algoritmos mem\'eticos para diseñar sus horarios de clases \cite{B Paechter and A Cumming and MG Norman and H Luchian}. Su codificación de horario especifica una lista de intervalos de tiempo propuestos para cada evento. Estos intervalos de tiempo son analizados en orden y el intervalo más conveniente es movido hacia el tope de la lista. El operador de recombinación construye una nueva lista de intervalos propuestos tomándolos de los padres correspondientes.

\subsubsection{Recocido Simulado}

Recocido simulado es una estrategia de búsqueda que sigue la pista de una solución factible. En cada iteración un vecino es generado, otro horario factible, ligero y aleatoriamente alterado por el horario actual. Este nuevo horario es aceptado si es mejor que el anterior. Si el nuevo horarario no es mejor este aún puede ser aceptado con una probabilidad relacionada al parámetro llamado temperatura. La temperatura y por tanto la probabilidad de que un vecino de menor calidad sea aceptado decrece con cada iteración o tras un número de iteraciones (este número puede ser constante o puede crecer a medida que la temperatura disminuya). Este proceso es análogo al proceso de enfriamiento de un recocido real. Un incoveniente del recocido simulado es que el proceso de enfriamiento puede tomar gran tiempo en aras de obtener buenos resultados.

En el TISSUE de Swansea se ha aplicado satisfactoriamente esta t\'ecnica en la generación de horarios para exámenes
\cite{J Thompson and KA Dowsland, J Thompson and KA Dowsland 2}. Los autores, Thompson y Dowsland usaron un modelo de coloración de grafos. Para simplificar el problema ellos lo dividieron en dos fases, la primera encontrar una solución factible y luego optimizar las restricciones suaves. Sin embargo esta t\'ecnica produce un espacio de búsqueda muy disperso, por lo que usan las cadenas de Kempe para lograr un espacio de búsqueda más conectado. Más que una simple función geom\'etrica de enfriamiento, Dowsland usa una función no monótona que aumenta la temperatura si un horario vecino es rechazado. Además la razón entre los vecinos rechazados y los aceptados incrementa a medida que progresa la búsqueda.

\subsubsection{Búsqueda Tab\'u}

Al igual que recocido simulado, búsqueda tabu mantiene solo un horario factible. La diferencia radica en cómo se acepta el movimiento hacia un horario vecino. Búsqueda tabu mantiene una lista de movimientos tabu, representando horarios que, han sido visitados recientemente pero que son prohibidos para evitar areas ya visitadas y así escapar de óptimos locales. La lista tabu usualmente tiene un tamaño fijo, los movimientos tabu más viejos son eliminados para dar espacio a nuevos movimientos tabu. Puede darse el caso que un movimiento tabu alcance una buena solución, es por eso que es mantenida la mejor solución encontrada hasta el momento. Si un horario tabu alcanza la mejor solución registrada entonces este horario es eliminado de la lista.

Búsqueda tabu ha sido aplicada satisfactoriamente por Boufflet y N\`egre para generar horarios de exámenes en la Universidad de Tecnología de Compi\`egne \cite{JP Boufflet and S Negre}. Su lista tabu contiene los siete últimos movimientos. Si de los vecinos del horario actual no existe uno mejor, entonces se selecciona un horario de la lista tabu.

Formulando el problema de los horarios como un problema de asignación, Hertz desarrolló y aplicó el algoritmo tabu TATI \cite{A Hertz}, el cual más tarde adaptó para resolver horarios más complejos y restringidos de la vida real \cite{A Hertz 2}. La duración de una clase no es fijada y existen diez tipos diferentes de movimientos (mover una clase a otro día, cambiar la duración de una clase, etc). Sin embargo, luego de un número de iteraciones es tabu mover una clase hacia su día original.

\subsubsection{Programación Lógica de Restricciones}

El problema de los horarios puede ser modelado como un Problema de Satisfacción de Restricciones (Constraint Satisfaction Problem, CSP). En un CSP a cada variable se le debe asignar un valor de su dominio y que la asignación cumpla con un conjunto de restricciones. Programación Lógica de Restricciones (Constraint Logic Programing, CLP) esta basada en la aplicación de lenguajes de programación de lógica declarativa como Prolog a CSPs. Un programa en Prolog consiste en un conjunto de clausulas que serian las restricciones y la satisfacibilidad de estas es chequeda en ejecución.

En CLP, una estrategia de etiquetado dicta el orden en que el espacio de búsqueda debe ser explorado, lo cual es vital para una búsqueda efectiva. Existen dos ordenamientos, el orden en que la variables deben ser asignadas y el orden en que los valores deben ser procesados.

Un sistema en Prolog para horarios de clases ha sido desarrollado por White en la Universidad de Ottawa \cite{L Kang and GM White, C Cheng and L Kang and N Leung and GM White}. Las restricciones son divididas en primarias y secundarias. Las restricciones primarias deben ser siempre cumplidas. Si no es posible asignar un aula o un espacio de tiempo a un encuentro sin violar alguna restricción primaria, entonces el programa retrocede, relajando restricciones secundarias una a una hasta que la asignación sea válida.

Boizumault, Delon y P\'eridy han diseñado un programa para horarios de exámenes \cite{P Boizumault and Y Delon and L Peridy} en CHIP, un lenguaje de programación lógica de restricciones basado en Prolog, el cual provee varios tipos de restricción. Proponen un tipo de restricción ``acumulativa'' que limita la cantidad de recursos que pueden ser usados en un tiempo, Boizumault la utiliza para implementar la restricción referente a la capacidad de las aulas. En su estrategia de búsqueda asigna primero los exámenes en los que intervienen un mayor número de estudiantes. Boizumault, Gu\'eret y Jussien han tambi\'en implementado un sistema para horarios de clases llamado FELIAC \cite{P Boizumault and C Gueret and N Jussien, C Gueret and N Jussien and P Boizumault and C Prins}. Las clases de gran audiencia son planificadas primero en los dias donde existan un menor número de clases planificadas. Relajación de restricciones es esencial para CSP con muchas restricciones como son los horarios. Los problemas en que las restricciones son relajadas son llamados dinámicos. Por cada asignación fallida, FELIAC guarda una ``justificación'' que identifica cuál restricción es violada por la asignación. Estas justificaciones son usadas para revertir los efectos de una restricción cuando es relajada.

\subsection{Modelando datos del problema}

El problema de los horarios puede variar considerablemente entre universidades, por lo tanto en aras de que pueda ser aplicable a diferentes instituciones, un sistema debe ser flexible en la manera en que sus restricciones y recursos son especificados.

Un módulo para horarios de universidad está siendo desarrollado por el Grupo de Planificación y Programación Automatizada de la Universidad de Nottingham, el cual incorpora un sistema altamente flexible en el que los objetos (dígase encuentros y recursos) pueden ser definidos y agrupados. Por simplicidad, la siguiente descripción trata sobre recursos, pero los encuentros pueden ser definidos y agrupados de la misma manera.

Cada tipo de recurso (ejemplo aulas) tiene en conjunto de atributos que permite definir recursos individuales. En la tabla \ref{table:atributes} se muestran cuatro atributos que las aulas pueden tener.

\begin{table}
	\caption{Conjunto de atributos para las aulas}
	\begin{center}
		\label{table:atributes}
		\begin{tabular}{lp{7.5cm}r}
			Nombre del atributo & Valores del atributo \\ \hline
			Proyector &\ Verdadero si el aula contiene un proyector, falso en otro caso \\
			Propósito &\ Usos que se le puede dar al aula, salón de conferencia, laboratorio \\
			Edificio &\ Edificio al que pertenece el aula \\
			Capacidad &\ Cantidad de estudiantes que el aula puede tener
		\end{tabular}
	\end{center}
\end{table}

El encargado de diseñar el horario puede adicionar, modificar o eliminar atributos de cualquier tipo con el objetivo de modificar la información que se tiene sobre los recursos. Existen dos formas de manipular los datos de los recursos:

\begin{description}
	\item[Mediante recursos:] Buscando o cambiando los valores de sus atributos. Ejemplo, el encargado pudiese querer saber a qu\'e edificio pertenece un local determinado, o a un edificio asignar un nuevo local.
	\item[Mediante conjuntos:] Teniendo que recursos pertenecen a un conjunto determinado y asi adicionar o eliminar recursos de dicho conjunto.
\end{description}

Este sistema de la Universidad de Nottingham provee una plataforma poderosa y flexible para la especificación de restricciones. Una restricción de asignación de recursos enlaza una lista de recurso a una lista de encuentros. Cada lista puede ser conformada por objetos individuales o conjuntos de objetos. Por ejemplo, una restricción de asignación de recursos pudiese especificar que todas las conferencias de una determinada asignatura tomen lugar en un local específico.

\subsection{Conclusiones}

La ardua tarea de crear un horario para una universidad puede ser facilitada considerablemente por la automatización. Sin embargo, dada la inmensa complejidad y variedad de este problema aún queda mucho camino por recorrer. Los algoritmos más exitosos hasta el momento han sido combinaciones de metaheurísticas generales con un alto grado de conocimiento heurístico específico para cada problema.

En los últimos cuarenta años ha habido un constante flujo de artículos relacionados con la optimización de horarios, aunque recientemente se ha incrementado considerablemente el inter\'es por este tema. Un indicador de este incremento es el \'exito que han tenido las conferencias de Practice and Theory of Automated Timetabling (PATAT 2010) en Northern Ireland.

Muchos investigadores de todo el mundo están desarrollando nuevas herramientas para automatizar este proceso, tambi\'en se ha publicado un gran volumen de resultados. Aún así, no existe un estándar que permita comparar los diferentes m\'etodos utilizados, aunque un grupo de universidades está trabajando en ello. Con las conferencias de PATAT establecidas como un evento regular parece que se desarrollará mucho trabajo en esta area.

\section{Problema de Satisfacción de Restricciones}

Esta sección tratará sobre los Problemas de Satisfacción de Restricciones, cuyo estados y objetivo conforman una representación muy sencilla, estándar y estructurada. Algoritmos de búsqueda pueden ser definidos para que tomen ventaja de la estructura de los estados y ser usados con un propósito general más que como una heurística específica del problema en cuestión.

\subsection{Definición}

Formalmente hablando, un Problema de Satisfacción de Restricciones (Constraint Satisfaction Problem, CSP) es definido por un conjunto de variables, $X_{1},X_{2},...,X_{n}$, y un conjunto de restricciones, $C_{1},C_{2},...,C_{m}$. Cada variable $X_{i}$ tiene un dominio $D_{i}$ no vacio de posibles valores. Cada restricción $C_{i}$ involucra algún subconjunto de las variables y especifica la combinación de valores válidas para ese subconjunto. Un estado del problema está definido por una asignación de valores a algunas o a todas las variables, $\{X_{i}=v_{i}, X_{j}=v_{j}, ...\}$. Una asignación que no viole ninguna restricción es llamada una asignación consistente o válida. Una asignación se dice completa si contiene a todas las variables, y una solución para un CSP es una asignación completa válida, o sea, que cumpla con todas las restricciones. Algunos CSPs tambi\'en requieren de una solución que maximice una función objetivo.

En la figura \ref{Australia}(a) se muestra un mapa de Australia con sus estados. Supongamos que tenemos la tarea de colorear cada estado con los colores rojo, verde o azul, de forma tal que estados vecinos no tengan el mismo color. Para formularlo como un CSP, definiremos que las variables sean los estados: WA, NT, Q, NSW, V , SA\ y\ T. El dominio de cada variable  es el conjunto \{rojo, verde, azul\}. La restricción dicta que regiones vecinas tengan colores distintos. Existen varias posibles soluciones , una de ellas: \{ WA = rojo, NT = verde, Q = rojo, NSW = verde, V = rojo, SA = azul, T = rojo \}. Es útil visualizar un CSP como un grafo de restricciones, como se muestra en la figura \ref{Australia}(b). Los nodos del grafo corresponden a las variables del problema y los arcos a las restricciones.

\begin{figure}
	\begin{center}
		\includegraphics[scale=0.5]{Graphics/Australia}
		\caption{(a) Principales estados y territorios de Australia. (b) El problema de coloración de mapas representado como un grafo de restricciones.}
		\label{Australia}
	\end{center}	
\end{figure}

Modelar un problema como un CSP confiere beneficios importantes, puesto que la representación de los estados en un CSP
conforman un patrón estándar, la función sucesor y el objetivo final pueden ser escritos de una forma gen\'erica que se aplique a todos los CSPs. Adem\'as, podemos desarrollar heur\'isticas gen\'ericas efectivas que no requieran conocimiento espec\'ifico del problema. Finalmente, la estructura de un grafo de restricciones puede ser usada para simplificar el proceso de búsqueda de una solución, en algunos casos puede reducir la complejidad exponencialmente.

Es f\'acil ver que para un CSP puede ser determinada una formulaci\'on incremental como un problema de búsqueda est\'andar de la siguiente forma:

\begin{description}
	\item[Estado inicial:] La asignaci\'on vacia \{\}, en la que todas las variables est\'an sin asignar.
	\item[Funci\'on sucesor:] Un valor es asignado a una variable sin asignar, previendo que no entre en conflicto con las variables ya asignadas.
	\item[Objetivo final:] La asignaci\'on actual es completa.
	\item[Costo de la trayectoria:] Un costo constante (ejemplo, 1) por cada paso.
\end{description}

Toda soluci\'on debe ser una asignaci\'on completa y por lo tanto aparece en la profundidad $n$ si existen $n$ variables. Adem\'as el \'arbol de búsqueda solo se extiende hasta la profundidad $n$. Por estas razones, algoritmos de b\'usqueda primero en profundidad (DFS) son popularmente utilizados para CSPs.

El tipo m\'as simple de CSP es cuando las variables tienen dominio discreto y finito. Problemas de coloraci\'on de mapas pertenecen a este tipo, al igual que el problema de las 8 reinas. Si el tamaño m\'aximo del dominio de cada variable es $d$, entonces el n\'umero posible de asignaciones completas es $O(d^{n})$, exponencial con respecto al n\'umero de variables. Dentro de este tipo de problemas est\'an los llamados CSP booleanos, cuyas variables pueden tomar solamente los valores \emph{true} o \emph{false}, un ejemplo es el problema 3SAT \cite{Carla P. Gomes}.

Las variables discretas pueden tener dominios infinitos, por ejemplo, el conjunto de los enteros. Supongamos que se planifican trabajos de construcci\'on, las variables ser\'ian la fecha de inicio de cada trabajo y los posibles valores n\'umeros enteros que indican la cantidad de dias que la tarea puede demorar. Para dominios infinitos no es posible describir restricciones enumerando todas las posibles combinaciones de valores. Es necesario utilizar un lenguaje de restricci\'on. Por ejemplo, si el trabajo $J_{1}$ que demora 5 dias debe concluir antes que comience $J_{3}$, entonces necesitariamos una desigualdad algebraica como $J_{1} + 5 < J_{3}$.

Problemas de satisfacci\'on de restricciones con dominios continuos son muy comunes en el mundo real y son ampliamente estudiados en el campo de investigaci\'on de operaciones. La categoria m\'as conocida de CSP con dominios continuos son los problemas de programaci\'on lineal, donde las restricciones son desigualdades lineales que forman una regi\'on convexa. Los problemas de programaci\'on lineal pueden ser resueltos en tiempo polinomial con respecto al n\'umero de variables.

Adem\'as de estudiar el tipo de variables, es \'util ver el tipo de restricci\'on. El tipo m\'as simple son las restricciones unarias, las cuales restringen el dominio de una solo variable. Toda restricci\'on unaria puede ser eliminada simplemente preprocesando el dominio de dicha variable eliminando cada valor que no cumpla con la restricci\'on. Una restricci\'on binaria relaciona dos variables. Por ejemplo, $SA \neq NSW$ es una restricci\'on binaria. Un CSP binario contiene solo restricciones binarias y puede ser representado como un grafo de restricciones, figura \ref{Australia}(b).

Todas las restricciones que hemos visto hasta el momento son llamadas restricciones fuertes, el no cumplimiento de alguna de ellas invalida por completo una posible soluci\'on. Muchos CSPs de la vida real incluyen restricciones d\'ebiles, indicando cuales soluciones son preferidas. Por ejemplo, para un horario universitario un profesor $X$ podr\'ia preferir dar clases en la mañana, sin embargo un horario que no cumpla esto sigue siendo v\'alido, lo que no ser\'ia \'optimo. Se le puede asociar una prioridad a cada restricci\'on d\'ebil. Con esta formulaci\'on, los CSPs con restricciones d\'ebiles pueden ser resueltos con m\'etodos de optimizaci\'on. 

\subsection{B\'usqueda con Backtracking}

En la secci\'on anterior se di\'o una formulaci\'on de CSPs como problemas de b\'usqueda. Supongamos que se aplica un BFS a un CSP gen\'erico. Notar\'a enseguida algo terrible, el factor de ramificaci\'on en el nivel superior es $nd$, porque cualquiera de los $d$ valores puede ser asignado en cada $n$ variables. En el pr\'oximo nivel el factor ser\'ia de $(n-1)d$, y as\'i por los $n$ niveles. Al final tendriamos un \'arbol con $n! \cdot d^{n}$ hojas, aunque solo existen $d^{n}$ posibles asignaciones completas.

Nuestra aparentemente razonable pero ingenua formulaci\'on del problema ignora una propiedad crucial com\'un a todos los CSPs: conmutatividad. Un problema es conmutativo si el orden de aplicar un conjunto de acciones no influye en el resultado. Cuando se tiene una asignaci\'on parcial no importa c\'omo se lleg\'o a ella, es decir, el orden en que se asignaron las variables no es determinante. Todos los algoritmos de b\'usqueda para CSPs generan sucesores considerando posibles asignaciones para solo una variable en cada nodo del \'arbol de b\'usqueda. Por ejemplo, en el nodo ra\'iz del \'arbol de b\'usqueda para la coloraci\'on del mapa de Australia, se debe escoger entre las posibles asignaciones $SA = rojo$, $SA = verde$ y $SA = azul$, pero nunca entre $SA = rojo$ y $WA = azul$. Con esta restricci\'on, el n\'umero de hojas es $d^{n}$, como se esperar\'ia.

El t\'ermino de b\'usqueda con backtracking es usado para una b\'usqueda primero en profundidad que escoge valores para una variable a la vez y retrocede (backtrack) cuando dicha variable no le quedan valores v\'alidos a asignar. El algoritmo se muestra en la figura \ref{DFS}. Producto de que la representaci\'on de un CSP est\'a estandarizada no es necesario proveer un estado inicial espec\'ifico del problema, como tampoco una funci\'on sucesor ni un objetivo final.

\begin{figure}
	\begin{center}
		\includegraphics[scale=0.65]{Graphics/DFS}
		\caption{Algoritmo de b\'usqueda con backtracking para CSPs.}
		\label{DFS}
	\end{center}	
\end{figure}

En el momento de escoger una variable a asignar podr\'ia preguntarse:

\begin{itemize}
	\item ¿Qu\'e variable se debe escoger para asignar y en qu\'e orden se deben procesar los valores de su dominio?
	\item ¿Cu\'ales ser\'an las consecuencias de la actual asignaci\'on en las variables sin asignar?
\end{itemize}

A cada una de estas preguntas se le dar\'a respuesta en las siguientes secciones.

\subsubsection{Ordenaci\'on de las Variables y sus Valores}

El algoritmo presentado anteriormente contiene la línea:

\begin{center}
	\scriptsize var $\leftarrow$ SELECT-UNASSIGNED-VARIABLE ( VARIABLE [ csp ] , assignment, csp).
\end{center}

Por defecto SELECT-UNASSIGNED-VARIABLE simplemente selecciona la pr\'oxima variable a asignar en el orden en que se definieron las variables inicialmente. Este orden est\'atico de seleccionar las variables rara vez resulta ser el m\'as eficiente. Existe una heur\'istica llamada \emph{Minimum Remaining Values (MRV)} que selecciona la variable con menos valores v\'alidos en su dominio. Esta heur\'istica es tambi\'en llamada \emph{Most Constrained Variable} o \emph{Fail-First}, esta \'ultima porque se escoge la variable con m\'as probabilidad de fallar, logrando una temprana poda del \'arbol. Si existe una variable $X$ que no le queden valores v\'alidos en su dominio, esta variable es seleccionada por la heur\'istica y el fallo es detectado inmediatamente, evitando asignar otras variables innecesariamente si al final cuando se trate de asignar $X$ se detectaria un fallo.

La heur\'istica MRV no nos ayuda a la hora de seleccionar la primera regi\'on a colorear del mapa de Australia, porque en un principio todos los dominios de valores tienen el mismo tamaño. En este caso, existe una heur\'istica llamada \emph{degree heuristic} que selecciona la variable que est\'a involucrada en la mayor cantidad de restricciones posibles con las dem\'as variables sin asignar. Siguiendo esta idea se seleccionaria la variable SA porque es la variable que aparece en m\'as restricciones.

Una vez que la variable haya sido seleccionada, el algoritmo debe decidir en que orden procesar los valores del dominio. La heur\'istica \emph{least constraining value} puede ser efectiva en algunos casos. La idea es procesar primero los valores que menos restrinjan el dominio de las variables vecinas en el grafo de restricciones. En general esta heur\'istica trata de mantener lo m\'as flexible las futuras asignaciones de variables. Si queremos encontrar todas las soluciones del problema, y no solamente la primera, entonces el orden de los valores no es importante porque habr\'ia que analizarlos todos.

\subsubsection{Propagaci\'on de Informaci\'on a trav\'es de las Restricciones}

Actualmente nuestro algoritmo solo considera las restricciones sobre una variable en el momento en que es seleccionada por el m\'etodo \emph{SELECT-UNASSIGNED-VARIABLE}. Si tenemos en cuenta estas restricciones antes de asignar dicha variable podemos reducir dr\'asticamente el espacio de b\'usqueda.

Una forma de hacer un mejor uso de las restricciones durante la b\'usqueda es mediante la t\'ecnica \emph{forward checking}. Cuando una variable $X$ es asignada, el proceso de forward checking examina cada variable $Y$ sin asignar que est\'e relacionada con $X$ en alguna restricci\'on, y elimina del dominio de $Y$ cualquier valor que sea inconsistente con el valor asignado a $X$.

Aunque forward checking detecte muchas inconsistencias, a veces deja de detectar algunas por no mirar adelante lo suficiente. \emph{Constraint propagation} es el t\'ermino general usado para la propagaci\'on de las implicaciones que puede tener una asignaci\'on en sus variables vecinas no asignadas.

La t\'ecnica \emph{consistencia de arco (arc consistency)} provee un m\'etodo r\'apido de propagaci\'on de restricci\'on que es sustancialmente mejor que forward checking. Con arcos, se refiere a los arcos dirigidos del grafo de restricciones. En el problema de colorear el mapa de Australia, el arco SA $\rightarrow$ NSW es consistente si por cada valor $x$ de SA, existe un valor $y$ de NSW que es consistente con $x$. Para ilustrar mejor, supongamos que los dominios de SA y NSW son $\{azul\}$ y $\{rojo, azul\}$ respectivamente. Para la asignaci\'on SA = azul exite una asignaci\'on consistente NSW = rojo, por lo que el arco SA $\rightarrow$ NSW es consistente. De la otra forma, el arco NSW $\rightarrow$ SA no es consistente, para la asignaci\'on NSW = azul no existe una asignaci\'on consistente para SA. Este arco puede hacerse consistente eliminando el valor azul del dominio de NSW.

El chequeo de consistencia de arco puede ser aplicado como un preprocesado antes de iniciar el proceso de b\'usqueda o como forward checking tras cada asignaci\'on durante el proceso de b\'usqueda. En cualquiera de los dos casos el proceso debe ser aplicado repetidamente hasta que no queden inconsistencias. Esto es, porque donde quiera que un valor es eliminado del dominio de alguna variable, pueden aparecer nuevas inconsistencias en los arcos que apuntan a esa variable. El algoritmo para consistencia de arco, \emph{AC-3} utiliza una cola para mantener almacenado todos los arcos que necesitan ser chequeados en b\'usquedas de posibles inconsistencias. Ver figura \ref{AC3}. Cada arco $(X_{i},X_{j})$ es sacado de la cola, si alg\'un valor es eliminado del dominio de $X_{i}$, entonces todo arco $(X_{k},X_{i})$ apuntando a $X_{i}$ es insertado en la cola para ser chequeado. La complejidad de este algoritmo puede ser analizada de la siguiente manera, un CSP binario tiene $O(n^{2})$ arcos, cada arco $(X_{k},X_{i})$ puede ser insertado en la cola a lo sumo $d$ veces, porque $X_{i}$ tiene m\'aximo $d$ valores y chequear la consistencia de un arco puede hacerse en $O(d^{2})$, por lo tanto el costo total del algoritmo es $O(n^{2}d^{3})$. Aunque es m\'as costoso que forward checking, vale la pena el costo extra.

\begin{figure}
	\begin{center}
		\includegraphics[scale=0.65]{Graphics/AC3}
		\caption{Algoritmo de consistencia de arco AC-3.}
		\label{AC3}
	\end{center}	
\end{figure}

Los problemas de satisfacci\'on de restricciones incluyen 3SAT como un caso especial, por lo tanto no podemos esperar encontrar un algoritmo que en tiempo polinomial nos diga si dado un CSP es consistente o no. Es por eso que el algoritmo presentado de consistencia de arco no detecta todas las inconsistencias que puedan existir. Existen otras formas de propagaci\'on de informaci\'on a trav\'es de restricciones m\'as fuertes que consistencia de arco, son llamadas k-consistencia.

\subsection{B\'usqueda Local para Problemas de Satisfacci\'on de Restricciones}

Algoritmos de b\'usqueda local suelen ser muy efectivos resolviendo CSPs. Usan asignaciones completas en todo momento, el estado inicial asigna un valor a cada variable, sin importar si se viola alguna restricci\'on, y la funci\'on sucesor usualmente trabaja cambiando el valor asignado a una variable en cada momento. Por ejemplo, en el problema de las 8 reinas, un estado inicial puede ser una reina en cada columna, y la funci\'on sucesor ser\'ia seleccionar una reina y moverla dentro de su columna.

En el momento de asignar un nuevo valor a una variable, una heur\'istica razonable ser\'ia seleccionar el valor que resulte en menos conflictos con las dem\'as variables. En la figura \ref{MinConflicts} se muestra el algoritmo y en la figura \ref{8queens} se ilustra una instancia del problema de las 8 reinas al que se le aplica dicho algoritmo.

\begin{figure}
	\begin{center}
		\includegraphics[scale=0.65]{Graphics/MinConflicts}
		\caption{Algoritmo MIN-CONFLICTS para resolver CSPs mediante b\'usqueda local.}
		\label{MinConflicts}
	\end{center}	
\end{figure}

\begin{figure}
	\begin{center}
		\includegraphics[scale=0.65]{Graphics/8queens}
		\caption{Aplicaci\'on del algoritmo MIN-CONFLICTS para una instancia del problema de las 8 reinas.}
		\label{8queens}
	\end{center}	
\end{figure}

El algoritmo MIN-CONFLICTS es sorprendentemente efectivo para varios CSPs, especialmente cuando comienza por un ``conveniente'' estado inicial. En un promedio de 50 pasos puede resolverse el problema de un mill\'on de reinas. Hablando a grandes rasgos, b\'usqueda local es buena para n-reinas porque las soluciones est\'an densamente distribuidas a trav\'es de todo el espacio.

Otra importante ventaja de b\'usqueda local es que puede ser usada para problemas que cambian din\'amicamente. Esto es particularmente para problemas de planificaci\'on de actividades. La planificaci\'on semanal de una aerol\'inea puede involucrar miles de vuelos y a decenas de miles de personas, pero un mal tiempo puede hechar por tierra toda la planificaci\'on hecha a priori. Ser\'ia recomendable replanificar todos los vuelos con el menor n\'umero de cambios. Esto puede lograrse f\'acilmente con una b\'usqueda local partiendo de la actual planificaci\'on. Una b\'usqueda con backtracking con nuevas restricciones requerir\'a de mucho tiempo y seguramente la soluci\'on encontrada tendr\'ia muchos cambios con respecto a la inicial.
