\chapter{Estado del Arte}

\section{Timetabling}

El diseño de un horario de clases o de exámenes para una universidad es una tarea bastante grande y compleja.
Existen diferentes departamentos y facultades, cada uno con sus propias ideas acerca de cómo y cuándo deben
comenzar sus cursos. Además, los estudiantes pueden tomar cursos de diferentes departamentos e incluso de
diferentes facultades. Numerosos sistemas de horarios para universidades han sido diseñados y se trabaja
en dirección a crear un estándar que permita comparar un sistema con otro. En esta sección se resumirán
las principales técnicas y trabajos recientes en cuanto a horarios, sea de clases o de exámenes, y servirá
como una introducción a este campo de investigación.

Un horario (timetable en Inglés) es una asignación de un conjunto de encuentros en un tiempo determinado.
Un encuentro es una combinación de recursos (aulas, personas o materiales de clase), algunos de ellos pueden
ser específicados por el problema y otros deben ser asignados como parte de la solución. La generación de un
horiario se conoce que pertenece a la clase de problemas llamada NP-completo, hasta ahora no se conoce un
algoritmo polinomial que pueda darle solución.

Existen algunas variaciones fundamentales para el problema de los horarios. La generación de un horario para
la universidad puede ser dividido en dos principales categorias, horario para clases u horario para exámenes.
Se diferencian principalmente en:

\begin{itemize}
	\item Los exámenes deben ser planificados de forma tal que un estudiante no tenga más de un exámen al
		mismo tiempo, pero las clases usualmente son planificadas antes que los estudiantes la matriculen.
	\item El espacio también es una restricción, diferentes exámenes pueden compartir la misma aula o un exámen
		puede necesitar más de un aula, pero en un aula solo puede ser impartida una clase a la vez.
\end{itemize}

Carter defiende con respecto al problema del horario de exámenes: \emph{el principal reto es programar los
exámenes para un periodo de tiempo limitado evitando conflictos y satisfaciendo un número de restricciones
colaterales}. \cite{Carter's summary} Con conflictos se refiere a que el horario requiera de un recurso
al mismo tiempo en lugares distintos. Las restricciones colaterales varian entre instituciones, y entre
horarios para clases y horarios para exámenes. Pueden existir una infinidad de restricciones colaterales.
Reto es la palabra apropiada para usar en estos casos.
Como Bloomfield y McSharry dicen: \emph{Dependiendo del tamaño de los departamentos y de la diversidad
de cursos que ofrecen, el tiempo requerido para diseñar un horario de clases puede ir desde una simple
tarde de trabajo hasta un mes de arduo trabajo.} \cite{Bloomfield and McSharry says}

\section{Problema de Satisfacción de Restricciones}
