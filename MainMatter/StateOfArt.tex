\chapter{Estado del Arte}

\section{Timetabling}

El diseño de un horario de clases o de exámenes para una universidad es una tarea bastante grande y compleja.
Existen diferentes departamentos y facultades, cada uno con sus propias ideas acerca de cómo y cuándo deben
comenzar sus cursos. Además, los estudiantes pueden tomar cursos de diferentes departamentos e incluso de
diferentes facultades. Numerosos sistemas de horarios para universidades han sido diseñados y se trabaja
en dirección a crear un estándar que permita comparar un sistema con otro. En esta sección se resumirán
las principales técnicas y trabajos recientes en cuanto a horarios, sea de clases o de exámenes, y servirá
como una introducción a este campo de investigación.

\subsection{Definición del problema}

Un horario (timetable en Inglés) es una asignación de un conjunto de encuentros en un tiempo determinado.
Un encuentro es una combinación de recursos (aulas, personas o materiales de clase), algunos de ellos pueden
ser específicados por el problema y otros deben ser asignados como parte de la solución. La generación de un
horiario se conoce que pertenece a la clase de problemas llamada NP-completo \cite{TB Cooper and JH Kingston},
hasta ahora no se conoce un algoritmo polinomial que pueda darle solución.

Existen algunas variaciones fundamentales para el problema de los horarios. La generación de un horario para
la universidad puede ser dividido en dos principales categorias, horario para clases y horario para exámenes.
Se diferencian principalmente en:

\begin{itemize}
	\item Los exámenes deben ser planificados de forma tal que un estudiante no tenga más de un exámen al
		mismo tiempo, pero las clases usualmente son planificadas antes que los estudiantes las matriculen.
	\item El espacio también es una restricción, diferentes exámenes pueden compartir la misma aula o un exámen
		puede necesitar más de un aula, pero en un aula solo puede ser impartida una clase a la vez.
\end{itemize}

Carter defiende con respecto al problema del horario de exámenes: ``\emph{el principal reto es programar los
exámenes para un período de tiempo limitado evitando conflictos y satisfaciendo un número de restricciones
colaterales}'' \cite{Carter's summary} Con conflictos se refiere a que el horario requiera de un recurso
al mismo tiempo en lugares distintos. Las restricciones colaterales varian entre instituciones, y entre
horarios para clases y horarios para exámenes. Pueden existir una infinidad de restricciones colaterales.
Reto es la palabra apropiada para usar en estos casos.
Como Bloomfield y McSharry dicen: ``\emph{Dependiendo del tamaño de los departamentos y de la diversidad
de cursos que ofrecen, el tiempo requerido para diseñar un horario de clases puede ir desde una simple
tarde de trabajo hasta un mes de arduo trabajo}''. \cite{Bloomfield and McSharry says}

El proceso de diseño de un horario se vuelve más difícil por el hecho de que involucra a muchas personas.
Romero indentificó tres grupos fundamentales cada uno con sus propios objetivos y necesidades. \cite{Romero}

\begin{itemize}
	\item La administración fija los estándares  mínimos que el horario debe cumplir. Por ejemplo, algunas
		universidades especifican que los estudiantes no tengan dos exámenes en períodos consecutivos.
	\item Los intereses de cada departamento son los más notables en el diseño de un horario. Cada departamento
		quiere que el horario sea consecuente con las asignaturas que imparte, asi como la necesidad de un
		aula o laboratorio en específico. En el contexto de exámenes, se quisiera que los exámenes más complicados
		sean los primeros, para asi tener más tiempo para su calificación.
	\item El tercer grupo involucrado son los estudiantes, a quienes solo le interesará la porción de horario
		referente a ellos. Dada la cantidad de estudiantes y la diversidad de criterios que estos puedan tener,
		se hace muy difícil determinar cual es el mejor horario para ellos. Muchos estudiantes prefieren no
		tener clases lo viernes y entre cada clases tener un descanso.		
\end{itemize}

Las restricciones para un horario pueden ser muchas y variadas. A continuación se presentan algunas de 
las más comunes:

\begin{description}
	\item[Asignación de recursos] Un recurso puede ser asignado a otro recurso de tipo diferente o a un encuentro.
		Por ejemplo, un profesor prefiere impartir sus clases en una determinada aula o un exámen debe ser realizado
		en un determinado edificio.
	\item[Asignación de tiempo] Un encuentro o un recurso debe ser asignado a un tiempo. Esta restricción puede
		ser usada para específicar días en los que un profesor no está disponible, o para pre-asignar un
		tiempo a un determinado encuentro.
	\item[Restricción de tiempo entre encuentros] Ejemplos comunes de esta clase de restricción son que un encuentro
		deba suceder antes de otro, o que un conjunto de exámenes deba realizarse simultaneamente.
	\item[Esparcimiento de encuentros] Los encuentros deben estar esparcidos en el tiempo. Por ejemplo, un
		estudiante no debe tener más de un exámen el mismo día.
	\item[Coherencia entre encuentros] Estas restricciones son diseñadas para producir un horario más organizado
		y conveniente. A menudo entran en contradicción con el esparcimiento de encuentros. Ejemplos de ellas es,
		que un profesor prefiera dar todos sus turnos en los primeros tres dias de la semana, dejándole los
		últimos dos dias libre.
	\item[Capacidad de las aulas] La cantidad de estudiantes no debe exceder la capacidad del aula.
	\item[Continuidad] Cualquier restricción cuyo propósito sea asegurar que el horario sea constante y predecible.
		Por ejemplo, conferencias de la misma asignatura deben ser programadas en la misma aula o en el mismo tiempo.
\end{description}

Las restricciones son usualmente dividadas entre las categorias fuertes y débiles:

\begin{description}
	\item[Restricciones fuertes] Un horario que no cumpla con alguna restricción fuerte no es una solución factible
		y debe ser modificado. Ejemplo, una persona no debe tener planificado más de un encuentro al mismo tiempo.
	\item[Restricciones débiles] Estas restricciones son menos importantes que las fuertes y es casi imposible
		cumplirlas todas. En algunas ocasiones se mide la calidad de un horario por el número de restricciones
		débiles que logra cumplir. Algunas restricciones débiles son más importantes que otras, esto puede ser
		específicado asociandoles una prioridad.
\end{description}

Descrita la complejidad del problema, una herramienta computacional tendria una gran utilidad en la realización
de esta complicada tarea.

\subsection{Encuesta realizada en universidades británicas}

El problema de los horarios puede variar enormente entre distintas universidades. El Grupo de Planificación
y Programación Automatizada de la Universidad de Nottingham lanzó una encuesta \cite{survey of University of Nottingham} sobre la planificación de exámenes en las universidades británicas. La encuesta fue enviada a
95 universidades, de ellas respondieron 56.

La encuesta mostró cuan distinto puede ser el problema de los horarios entre las diferentes instituciones.
Por ejemplo, el número de exámenes a planificar puede variar desde unos cientos hasta medio millar por sesión,
y los estudiantes involucrados pueden ser desde unos quinientos hasta unos veinte mil. Como dijo un
encargado de programar el horario, ``\emph{Existen muchas variaciones en cuanto a la generación de un horario,
podriamos escribir un libro}''.

Uno de los mayores problemas para las universidades es encontrar locales en los que realizar los exámenes.
Varias universidades resuelven el problema de la insuficiencia de aulas alquilando locales externos a la
universidad. Esto puede resolver el problema en parte, pero si estos locales son utilizados en otras funciones
su disponibilidad se vuelve impredecible. También hay que considerar que estos locales no siempre están cerca de
la universidad, causando que los estudiantes necesiten más tiempo para trasladarse de un local a otro.

De las universidades que respondieron la encuesta, sorprende que el 42\% no se apoyan en ninguna herramienta
computacional para realizar este proceso. Una posible razón para esto es que en varias universidades el
horario de un curso no cambia significativamente con respecto al anterior. Aquellas universidades que no
se apoyan en el horario del curso anterior ni en alguna herramienta computacional demoran al menos cuatro
semanas en diseñar su horario.

Otra razón para la falta de automatización en este proceso es producto de que los encargados de diseñar el horario
algunas veces desconfian de los beneficios que esto podria traer. El departamento de exámenes de una universidad
dijo que ellos podian ``\emph{ver las desventajas de un sistema completamente automatizado. Preferimos trabajar
desde la etapa inicial del diseño con una tabla y un lapiz. Esto permite una vista general del progreso de la
generación mejor que un sistema computacional basado en preguntas.}''

\begin{figure}
	\begin{center}
		\includegraphics{Graphics/computer_usage}
		\caption{Automatización del proceso de diseño de horarios en universidades británicas.}
	\end{center}	
\end{figure}

De las universidades que respondieron la encuesta el 21\% genera su horario de forma automática por
computadoras, aunque confiesan que a veces es necesario la intervención manual con el objetivo de satisfacer
algunas particularidades que escapan del alcance de las computadoras. Algunas universidades desarrollan
sus propias herramientas y otras universidades adaptan las herramientas comerciales a sus necesidades.

Exiten dos metodologías fundamentales que utilizan aquellas universidades que no automatizan el proceso de
generación de horarios:

\begin{itemize}
	\item Permitir a cada facultad o departamento diseñar su propuesta de horario y luego la administración
		central los mezcla y elabora el horario final. Esto asume que las facultades y los departamentos
		son independientes entre ellos. Sin embargo, muchas universidades británicas están adoptando una
		organización modular, lo que implica que esta metodología no es siempre válida.
	\item Construir el horario de exámenes modificando el horario de clases. Una institución que casualmente
		estaba desarrollando un nuevo sistema, dijo: ``\emph{Esto, por supuesto no está del todo bien, porque
		pueden haber varias conferencias en diferentes intervalos de tiempo o el número de estudiantes involucrados
		es mayor a la capacidad de las aulas de exámenes}''. Esta metodología depende de la naturaleza del curso
		asociado al horario.
\end{itemize}

La encuesta demostró la gran necesidad de automatizar este proceso. Habiendo dicho esto, cualquier sistema
para que sea útil debe satisfacer la gran cantidad de criterios disimiles que existen entorno a este problema.
El principal problema es que el sistema debe ser capaz de generar horarios de gran calidad a pesar de las enormes
variaciones de este proceso encontradas en las diferentes universidades encuestadas. El sistema debe ser compatible
con trabajos hechos con anterioridad, debe ser fácil de usar y satisfacer todas las necesidades de los departamentos y facultades.

\subsection{¿Cómo comparar la calidad de las soluciones?}

Varios métodos han sido desarrollados y utilizados con éxito para solucionar el problema de los horarios.
\cite{VA Bardadym, MW Carter, MW Carter and G Laporte, JH Kingston}. Sin embargo, muchos de estos métodos 
solo son usados en un departamento o facultad en específico, rara vez son comparados entre sí. Una 
comparación precisa es vital para determinar cuál de los métodos computacionales es el mejor para
distintas instancias del problema. Cuando la técnica usada es relativamente simple, se ajusta con facilidad
a condiciones distintas. Con el paso del tiempo las técnicas se vuelven más sofisticadas, con el objetivo
de satisfacer nuevas necesidades.

No es fácil expresar claro y precisamente los requerimientos para un horario real. Los requerimientos básicos
como ``cada clase debe tener un profesor'' o ``nadie debe estar en dos lugares distintos al mismo tiempo'' son
fáciles de expresar, pero las ``restricciones colaterales'' como el aula ideal para impartir una asignatura o
el orden cronológico entre clases son muy difíciles de representar y por consiguiente de leer y entender. Esa
es la razón por qué es tan difícil adaptar un sistema hecho para otra institución a nuestras necesidades.

Un proyecto en desarrollo por la Universidad de Nottingham conjunto con la Universidad de Napier, la Universidad
de Reading, la Universidad de Sydney y la Universidad de Toronto intenta encontrar una forma de escribir cualquier
requerimiento, usando fórmulas lógicas. Esto permitirá utilizar herramientas desarrolladas por otras entidades tan
solo redefiniendo nuestros propios requerimientos.

\subsection{Técnicas para automatizar el diseño de un horario}

Para instancias del problema heurísticas específicas pueden dar buenos resultados, pero con el devenir del tiempo
los horarios de universidades se han vuelto más complejos. Actualmente hay una tendencia que fue confirmada
con las presentaciones de la primera conferencia internacional de ``Practice And Theory of Automated Timetabling''
\cite{D Abramson and J Abela}, de resolver el problema con algoritmos más generales, o metaheurísticas, como
recocido simulado, algoritmos evolutivos y búsqueda tabú. Las herísticas específicas pueden ser utilizadas
para disminuir el número de posibles soluciones a procesar o para optimizar localmente una solución. Programación
Lógica de Restricciones (Constraint Logic Programming) es también otra técnica utilizada.

\subsubsection{algoritmos genéticos}

Los algoritmos genéticos son análogos a la evolución propuesta por Darwin. Una población de horarios válidos
es mantenida. Los mejores horarios son seleccionados para formar las bases de la próxima generación, mejorando
la calidad de la población al mismo tiempo que se mantiene la diversidad.

La representación genética más común para un horario es una larga cadena de bits codificando cuándo y dónde
debe ocurrir un encuentro. Siendo así, los pares de los horarios seleccionados puden ser cruzados cortando
la cadena y pasando a sus descendientes información de ambos padres. No obstante, Corne, Ross y Fang proponen
una conveniente operación de mutación para ser más exitoso el cruzamiento \cite{D Corne and P Ross and HL Fang}.
Su sistema, GATT, está siendo usado para generar el horario de la Universidad de Edinburgh y algunas otras
instituciones.

\begin{figure}
	\begin{center}
		\includegraphics[scale=0.5]{Graphics/genetic_algorithms}
		\caption{Etapas de un algoritmo genético.}
	\end{center}	
\end{figure}

Paechter y Cumming desarrollaron ``Neeps and Tatties'' un sistema que se ha estado usando en el departamento
de Ciencia de la Computación de la Universidad de Napier. Su algoritmo genético codifica un horario como una ordenación
de eventos, que debe ser introducido a un programa especial que utiliza ese orden para producir un horario
\cite{B Paechter* and A Cumming* and H Luchian}. Este algoritmo necesita un operador de permutación 
que cambie el orden de los eventos heredados por el padre.

El Grupo de Planificación y Programación Automatizada de la Universidad de Nottingham ha estado desarrollando
algoritmos genéticos para horarios de exámenes utilizando un alto grado de conocimiento heurístico, para generar
una población inicial y para los operadores genéticos \cite{EK Burke and DG Elliman and RF Weare 1, 
EK Burke and DG Elliman and RF Weare 2, EK Burke and DG Elliman and RF Weare 3}. El operador de cruzamiento
trabaja a nivel de período, tomando encuentros planificados de ambos padres primero, y luego planificando
otros de acuerdo a la heurística de ordenamiento, hasta que no se puedan planificar más encuentros. Este tipo de operador de cruzamiento permite un uso eficiente de las aulas porque siempre trata de asignar un encuentro dondequiera que sea posible.

\subsubsection{algoritmos mem\'eticos}

Los algoritmos mem\'eticos son una extensión de los algoritmos gen\'eticos, basados en un modelo de cómo las ideas evolucionan. La unidad básica de las ideas son los memes, los que a diferencia de los genes, pueden ser mejorados durante su ciclo de vida. Es por eso que un simple algoritmo de búsqueda local (hill-climbing) es usado por intervalos
para asegurarse que todos los horarios miembros de la población son óptimos locales.

La Universidad de Nottingham se encuentra desarrollando algoritmos mem\'eticos para horarios de exámenes \cite{EK Burke and JP Newall and RF Weare}. Un operador de mutación es aplicado a los miembros seleccionados de la población. Despues de mutar cada miembro seleccionado este es optimizado por un algoritmo de optimización local. Esta combinación genera soluciones con buena calidad.

La Universidad de Napier está trabajando con algoritmos mem\'eticos para diseñar sus horarios de clases \cite{B Paechter and A Cumming and MG Norman and H Luchian}. Su codificación de horario especifica una lista de intervalos de tiempo propuestos para cada evento. Estos intervalos de tiempo son analizados en orden y el intervalo más conveniente es movido hacia el tope de la lista. El operador de recombinación construye una nueva lista de intervalos propuestos tomándolos de los padres correspondientes.

\subsubsection{recocido simulado}

Recocido simulado es una estrategia de búsqueda que sigue la pista de una solución factible. En cada iteración un vecino es generado, otro horario factible, ligero y aleatoriamente alterado por el horario actual. Este nuevo horario es aceptado si es mejor que el anterior. Si el nuevo horarario no es mejor este aún puede ser aceptado con una probabilidad relacionada al parámetro llamado temperatura. La temperatura y por tanto la probabilidad de que un vecino de menor calidad sea aceptado decrece con cada iteración o tras un número de iteraciones (este número puede ser constante o puede crecer a medida que la temperatura disminuya). Este proceso es análogo al proceso de enfriamiento de un recocido real. Un incoveniente del recocido simulado es que el proceso de enfriamiento puede tomar gran tiempo en aras de obtener buenos resultados.

En el TISSUE de Swansea se ha aplicado satisfactoriamente esta t\'ecnica en la generación de horarios para exámenes
\cite{J Thompson and KA Dowsland, J Thompson and KA Dowsland 2}.

\section{Problema de Satisfacción de Restricciones}
