\chapter{Estado del Arte}

\section{Timetabling}

El diseño de un horario de clases o de exámenes para una universidad es una tarea bastante grande y compleja.
Existen diferentes departamentos y facultades, cada uno con sus propias ideas acerca de cómo y cuándo deben
comenzar sus cursos. Además, los estudiantes pueden tomar cursos de diferentes departamentos e incluso de
diferentes facultades. Numerosos sistemas de horarios para universidades han sido diseñados y se trabaja
en dirección a crear un estándar que permita comparar un sistema con otro. En esta sección se resumirán
las principales técnicas y trabajos recientes en cuanto a horarios, sea de clases o de exámenes, y servirá
como una introducción a este campo de investigación.

Un horario (timetable en Inglés) es una asignación de un conjunto de encuentros en un tiempo determinado.
Un encuentro es una combinación de recursos (aulas, personas o materiales de clase), algunos de ellos pueden
ser específicados por el problema y otros deben ser asignados como parte de la solución. La generación de un
horiario se conoce que pertenece a la clase de problemas llamada NP-completo, hasta ahora no se conoce un
algoritmo polinomial que pueda darle solución.

Existen algunas variaciones fundamentales para el problema de los horarios. La generación de un horario para
la universidad puede ser dividido en dos principales categorias, horario para clases u horario para exámenes.
Se diferencian principalmente en:

\begin{itemize}
	\item Los exámenes deben ser planificados de forma tal que un estudiante no tenga más de un exámen al
		mismo tiempo, pero las clases usualmente son planificadas antes que los estudiantes la matriculen.
	\item El espacio también es una restricción, diferentes exámenes pueden compartir la misma aula o un exámen
		puede necesitar más de un aula, pero en un aula solo puede ser impartida una clase a la vez.
\end{itemize}

Carter defiende con respecto al problema del horario de exámenes: "\emph{el principal reto es programar los
exámenes para un período de tiempo limitado evitando conflictos y satisfaciendo un número de restricciones
colaterales}". \cite{Carter's summary} Con conflictos se refiere a que el horario requiera de un recurso
al mismo tiempo en lugares distintos. Las restricciones colaterales varian entre instituciones, y entre
horarios para clases y horarios para exámenes. Pueden existir una infinidad de restricciones colaterales.
Reto es la palabra apropiada para usar en estos casos.
Como Bloomfield y McSharry dicen: "\emph{Dependiendo del tamaño de los departamentos y de la diversidad
de cursos que ofrecen, el tiempo requerido para diseñar un horario de clases puede ir desde una simple
tarde de trabajo hasta un mes de arduo trabajo}". \cite{Bloomfield and McSharry says}

El proceso de diseño de un horario se vuelve más difícil por el hecho de que involucra a muchas personas.
Romero indentificó tres grupos fundamentales cada uno con sus propios objetivos y necesidades. \cite{Romero}

\begin{itemize}
	\item La administración fija los estándares  mínimos que el horario debe cumplir. Por ejemplo, algunas
		universidades especifican que los estudiantes no tengan dos exámenes en períodos consecutivos.
	\item Los intereses de cada departamento son los más notables en el diseño de un horario. Cada departamento
		quiere que el horario sea consecuente con las asignaturas que imparte, asi como la necesidad de un
		aula o laboratorio en específico. En el contexto de exámenes, se quisiera que los exámenes más complicados
		sean los primeros, para asi tener más tiempo para su calificación.
	\item El tercer grupo involucrado son los estudiantes, a quienes solo le interesará la porción de horario
		referente a ellos. Dada la cantidad de estudiantes y la diversidad de criterios que estos puedan tener,
		se hace muy difícil determinar cual es el mejor horario para ellos. Muchos estudiantes prefieren no
		tener clases lo viernes y entre cada clases tener un descanso.		
\end{itemize}

Las restricciones para un horario pueden ser muchas y variadas. A continuación se presentan algunas de 
las más comunes:

\begin{description}
	\item[Asignación de recursos] Un recurso puede ser asignado a otro recurso de tipo diferente o a un encuentro.
		Por ejemplo, un profesor prefiere impartir sus clases en una determinada aula o un exámen debe ser realizado
		en un determinado edificio.
	\item[Asignación de tiempo] Un encuentro o un recurso debe ser asignado a un tiempo. Esta restricción puede
		ser usada para específicar días en los que un profesor no está disponible, o para pre-asignar un
		tiempo a un determinado encuentro.
	\item[Restricción de tiempo entre encuentros] Ejemplos comunes de esta clase de restricción son que un encuentro
		deba suceder antes de otro, o que un conjunto de exámenes deba realizarse simultaneamente.
	\item[Esparcimiento de encuentros] Los encuentros deben estar esparcidos en el tiempo. Por ejemplo, un
		estudiante no debe tener más de un exámen el mismo día.
	\item[Coherencia entre encuentros] Estas restricciones son diseñadas para producir un horario más organizado
		y conveniente. A menudo entran en contradicción con el esparcimiento de encuentros. Ejemplos de ellas es,
		que un profesor prefiera dar todos sus turnos en los primeros tres dias de la semana, dejándole los
		últimos dos dias libre.
	\item[Capacidad de las aulas] La cantidad de estudiantes no debe exceder la capacidad del aula.
	\item[Continuidad] Cualquier restricción cuyo propósito sea asegurar que el horario sea constante y predecible.
		Por ejemplo, conferencias de la misma asignatura deben ser programadas en la misma aula o en el mismo tiempo.
\end{description}

Las restricciones son usualmente dividadas entre las categorias fuertes y débiles:

\begin{description}
	\item[Restricciones fuertes] Un horario que no cumpla con alguna restricción fuerte no es una solución factible
		y debe ser modificado. Ejemplo, una persona no debe tener planificado más de un encuentro al mismo tiempo.
	\item[Restricciones débiles] Estas restricciones son menos importantes que las fuertes y es casi imposible
		cumplirlas todas. En algunas ocasiones se mide la calidad de un horario por el número de restricciones
		débiles que logra cumplir. Algunas restricciones débiles son más importantes que otras, esto puede ser
		específicado asociandoles una prioridad.
\end{description}

Descrita la complejidad del problema, una herramienta computacional tendria una gran utilidad en la realización
de esta complicada tarea.

El problema de los horarios puede variar enormente entre distintas universidades. El Grupo de Planificación
y Programación Automatizada de la Universidad de Nottingham lanzó una encuesta \cite{survey of University of Nottingham} sobre la planificación de exámenes en las universidades británicas. La encuesta fue enviada a
95 universidades, de ellas respondieron 56.

La encuesta mostró cuan distinto puede ser el problema de los horarios entre las diferentes instituciones.
Por ejemplo, el número de exámenes a planificar puede variar desde unos cientos hasta medio millar por sesión,
y los estudiantes involucrados pueden ser desde unos quinientos hasta unos veinte mil. Como dijo un
encargado de programar un horario, "\emph{Existen muchas variaciones en cuanto a la generación de un horario,
podriamos escribir un libro}".

Uno de los mayores problemas para las universidades es encontrar locales en los que realizar los exámenes.
Varias universidades resuelven el problema de la insuficiencia de aulas alquilando locales externos a la
universidad. Esto puede resolver el problema en parte, pero si estos locales son utilizados en otras funciones
su disponibilidad se vuelve impredecible. También hay que considerar que estos locales no siempre están cerca de
la universidad, causando que los estudiantes necesiten más tiempo para trasladarse de un local a otro.

De las universidades que respondieron la encuesta, sorprende que el 42\% no se apoyan en ninguna herramienta
computacional para realizar este proceso. Una posible razón para esto es que en varias universidades el
horario de un curso no cambia significativamente con respecto al anterior. Aquellas universidades que no
se apoyan en el horario del curso anterior ni en alguna herramienta computacional demoran al menos cuatro
semanas en diseñar su horario.

Otra razón para la falta de automatización en este proceso es producto de que los encargados de diseñar el horario
algunas veces desconfian de los beneficios que esto podria traer. El departamento de exámenes de una universidad
dijo que ellos podian "\emph{ver las desventajas de un sistema completamente automatizado. Preferimos trabajar
desde la etapa inicial del diseño con una tabla y un lapiz. Esto permite una vista general del progreso de la
generación mejor que un sistema computacional basado en preguntas.}"

\includegraphics[scale=1]{Graphics/computer_usage}pepep

\section{Problema de Satisfacción de Restricciones}
